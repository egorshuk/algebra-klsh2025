\begin{enumerate}
    \item Сколько элементов в группе $D_5$? А сколько элементов порядка $2$?
    \item Докажите, что $A_n$ --- подгруппа группы $S_n$.
    \item Верно ли, что подгруппа абелевой группы всегда абелева? Если да — объясните. Если нет — приведите контрпример.
    \item В группе $D_6$ найдите композицию $r \circ s_1$, где $r$ --- поворот на $60^\circ$, а $s_1$ --- отражение относительно вертикальной оси.
    \item Напишите таблицу Кэли для группы $S_3$. Какие из элементов коммутируют между собой?
    \item Докажите, что в группе $D_n$ выполняется равенство: \[
            s \circ r = r^{-1} \circ s.
        \]
        Где $r$ --- поворот на $\rfrac{360^\circ}{n}$, а $s$ --- отражение относительно любой оси.
    % \item Докажите, что в группе порядка 15 не может быть подгруппы порядка 4.
    % \item Пусть порядок группы равен 12. Докажите, что порядок любой подгруппы делит 12. Может ли в этой группе существовать элемент порядка 7?
    % \item Докажите, что все элементы порядка $2$ и единица в группе $S_n$ образуют подгруппу. 
    \item Пусть $H = (\{-1, 1\}, \times)$ в $(\R^*, \times).$\footnote{Здесь ,,звёздочка`` обозначает то, что нет нуля.} Является ли $H$ подгруппой? Является ли $H$ абелевой?
    \item Изоморфны ли группы
        \begin{inumerate}
        \item $S_2$ и $\Z/(2)$
        \item $S_3$ и $\Z/(3)$
        \item $D_4$ и $S_4$
        \item $S_4$ и $D_{12}$
        \item[(f*)] $S_5$ и $D_{10}$
        \end{inumerate}.
    \item Найдите все подгруппы в 
        \begin{inumerate}
        \item $\Z/(6)$
        \item $S_3$
        \item $D_4$
        \item $D_6$
        \item[(e*)] $D_{12}$
        \item[(f*)] $A_4$
        \end{inumerate}.
    Для каждой подгруппы проверьте, что ее порядок делит порядок всей группы. Подумайте над тем, каким группам изоморфны каждая из них.
\item Летнешкольников заставили выложить плац правильной шестиугольной плиткой\footnote{Причем плитка самая обычная, на ней даже узоров никаких нет.}. Сколько существует симметрий такого замощения плиткой? Образуют ли они группу?
        Если да, то какой у нее порядок?
    \item Пусть $H$ --- множество всех перестановок из $S_4$, которые оставляют
        тройку на месте. Является ли $H$ подгруппой группы $S_3$?
        Если да, то какой у нее порядок и является ли она абелевой?
    \item Является ли множество $G = \{2^n \mid n \in \Z\}$ с операцией умножения группой?
        Если да, то является ли она абелевой? Какие в ней подгруппы?
    \item Придумайте свой объект, например, букву ,,Ж``. Опишите его группу симметрий. Подумайте, какой группе она изоморфна.
    \item Пусть $H$ подгруппа группы $G$. Тогда \emph{левым смежным классом} называется \(gH =\{gh \mid h \in H\}.\)
        \begin{enumerate}[(a)]
        \item Докажите, что два смежных класса либо совпадают, либо не пересекаются.
        \item[(b)*] Докажите, что все смежные классы находятся в биекции друг с другом.
        \addtocounter{enumii}{1}
        \item В каком соотношении находится порядок группы $H$, число смежных классов и порядок группы $G$?
        \item Почему в группе порядка $15$ не может быть подгруппы порядка $4$?
        \end{enumerate}
\end{enumerate}
