\begin{enumerate}
    \item Докажите, что $A_n$ -- подгруппа группы $S_n$.
    \item Верно ли, что подгруппа абелевой группы всегда абелева? Если да — объясните. Если нет — приведите контрпример.
    \item Докажите, что в группе порядка 15 не может быть подгруппы порядка 4.
    \item Пусть порядок группы равен 12. Докажите, что порядок любой подгруппы делит 12.
        Может ли в этой группе существовать элемент порядка 7?
    \item Докажите, что все элементы порядка $2$ и единица в группе $S_n$ образуют подгруппу. 
    \item Пусть $H = (\{-1, 1\}, \times)$ в $(\R^*, \times).$\footnote{
        Здесь ``звёздочка'' обозначает то, что нет нуля.}
        Является ли $H$ подгруппой?
        Является ли $H$ абелевой?
    \item Найдите все подгруппы группы $S_3$. Укажите их порядок, проверьте, что
        их порядок делит порядок всей группы. Какие из подгрупп абелевы?
    \item Найдите все подгруппы группы $\Z/(6)$. Укажите их порядок, проверьте, что
        их порядок делит порядок всей группы. Какие из подгрупп абелевы?
    \item Летнешкольников заставили выложить плац правильной шестиугольной плиткой. 
        Сколько существует симметрий такого замощения плиткой? Образуют ли они группу?
        Если да, то какой у нее порядок?
    \item Пусть $H$ -- множество всех перестановок из $S_3$, которые оставляют
        тройку на месте. Является ли $H$ подгруппой группы $S_3$?
        Если да, то какой у нее порядок и является ли она абелевой?
    \item Множество $G = \{2^n \mid n \in \Z\}$ с операцией умножения является ли группой?
        Если да, то является ли она абелевой? Какие в ней подгруппы?
\end{enumerate}
