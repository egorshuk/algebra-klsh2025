\begin{enumerate}
    \item Даны множества $A = \{\{1, 2 \}, 3, 4 \}$, $B = \{1, 2, \{3 \}, 4\}$ и $C = \{1, 2, 3, 4\}.$ 
        \begin{inumerate}
        \item какие множества являются подмножествами других
        \item количество элементов каждого множества
        \item количество подмножеств каждого множества
        \item найлите $A \cup B$ 
        \item найдите $(A \cap B) \cup C$
        \item доказать, что $(A \cup C) \cap B = (A \cap B) \cup (C \cap B)$
        \item доказать, что $A \setminus (B \cup C) = (A \setminus B) \cap (A \setminus C)$.
        \end{inumerate}
    \item Нарисуйте все отображения $\{0, 1, 2\} \to \{0, 1\}$ и $\{0, 1\} \to \{0, 1, 2\}$. Сколько среди них сюръекций и инъекций?
    \item Из отображений \begin{inumerate}
        \item $\N \to \N \colon x \mapsto x^2$
        \item $\Z \to \Z \colon x \mapsto x^2$ 
        \item $\Z \to \Z \colon x \to 7x $
    \end{inumerate} выделите все вложения, наложения и биекции.
    \item Найдите слои отображения $\Z \to \Z, x \mapsto x^4$ над точками $0$ и $1$.
    \item Отображение $\N \to \N, x \mapsto |x| + 1$. Найдите полный прообраз точки $5$.  
    \item Покажите, что если $A \subset B$, то $A \times A \subset B \times B$. 
    \item На множестве $\Z$ задано отношение $x \sim y \iff x + y$ кратно 2. Докажите, что это является отношением эквивалентности. Найдите классы эквивалентности для чисел $0$, $1$, $5$, $-3$.
    \item Рассмотрите множество чисел от 1 до 50. Определите классы эквивалентности так, чтобы два числа были эквивалентны, если они имеют одинаковое количество простых делителей. Сколько всего будет классов эквивалентности? Укажите по одному представителю от каждого класса. 
    \item Cколько имеется таких отображений из пятиэлементного множества в двухэлементное, чтобы у каждой точки было не менее двух прообразов.
    \item Из отображений \begin{inumerate}
        \item $\Z/(12) \to \Z/(12), x \mapsto 2x$ 
        \item $\Z/(12) \to \Z/(12), x \mapsto 3x$
        \item $\Z/(12) \to \Z/(12), x \mapsto 7x$
    \end{inumerate} выделите все инъекии, сюръекции, а также биекции. Везде найдите образ. А также прообразы каждой точки, проверьте, что множество $\Z/(12)$ распадается в дизъюнктное объединение слоёв.
    \item Фиксируем $m, n \in \N$. Сколько всего имеетcя отображений $\{1, 2, \ldots, m\} \to \{1, 2, \ldots, n\}$ \begin{inumerate}
        \item произвольных
        \item биективных
        \item возрастающих
        \item инъективных
        \item неубывающих
        \item сюръективных неубывающих
        \item сюръективных
    \end{inumerate}.
\item Про каждое из следующим множеств выясните, существует ли биекция из первого во второе\footnote{Ноль здесь является натуральным числом.} 
    \begin{enumerate*}
        \item множество натуральных чисел;
        \item множество чётных натуральных чисел;
        \item множество натуральных чисел без числа $3$.
    \end{enumerate*}
\item На квадратном листе $[0;1]^2$ введено отношение, где $(a, b) \sim (c, d)$, если $|a-c| = 1$ и $b + d = 1$. Докажите, что это является отношением эквивалетности, а также назовите получившуюся поверхность.
\end{enumerate}
