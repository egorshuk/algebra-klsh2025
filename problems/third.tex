\begin{enumerate}
    \item Придумайте свою фигуру, например букву <<Ж>> или прямую пятиугольную призму, найдите её полную и собственную группу движений. 
    \item Напишите \emph{таблицу Кэли (умножения)} для групп $D_3, D_4, D_5$. Изоморфны ли данные группы каким-нибудь вам известным?
    \item Среди несобственных движений тетраэдра есть повороты на $180^\circ$ вокруг осей, проходящих через середины противоположных рёбер. Докажите, что такие движение образуют группу, изоморфную $V_4$.
    \item Докажите или опровергните
        \begin{inumerate}
            \item $D_2 \cong \Z/(2) \times \Z/(2)$   
            \item $D_3 \cong \Z/(2) \times \Z/(3)$
            \item $D_3 \cong A_3 \times \Z/(2)$
            \item $D_6 \cong A_4$
            \item $O_{\text{октаэдр}} \cong O_{\text{куб}}$.
        \end{inumerate}
    \item Проверьте, что полные и собственные группы куба, октаэдра и икосаэдра соотсветнно состоят из $48$ и $24$, $48$ и $24$, $120$ и $60$ движений.
    \item В собственной группе движений тетраэдра найдите для заданной вершины $V$ найдите две величины:  
        \begin{enumerate*}
            \item количество вершин, в которые можно преобразовать вершину $V$ каким-то движением $g$;
            \item количество преобразований, которые оставляют вершину $V$ неподвижной.
        \end{enumerate*}
        Проверьте, что произведение этих чисел дает порядок всей группы.

    \item Классифицируйте все повороты додекаэдра. Сколько их? Сколько в каждом классе?

    \item Докажите, что $SO_{\text{тетраэдр}}$ является подгруппой $SO_{\text{куб}}$.
    \item[9*.] Докажите, что $SO_{\text{куб}} \cong O_{\text{тетраэдр}}$.
\end{enumerate}
