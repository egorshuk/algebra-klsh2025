\begin{enumerate}
    \item Сколькими разными способами можно раскрасить грани тетраэдра в $n$ цветов, где две раскраски считаются \emph{одинаковыми}, если одну можно получить поворотов другой.
    \item Сколькими различными способами можно составить ожерелье из $n$ цветов, в котором будет \begin{enumerate}
        \item 5 бусин;
        \item 6 бусин;
        \item 7 бусин.
    \end{enumerate}

    \item Флаг некоторой страны состоит из трёх горизонтальных полос. Сколькими способами можно раскрасить его в $n$ цветов. Если флаги, отличающиеся перестановкой полос, считаются одинаковыми.

    \item Сколькими разными способами можно раскрасить рёбра куба в $n$ цветов, где две раскраски считаются \emph{одинаковыми}, если одну можно получить поворотов другой.

    \item Сколькими способами можно раскрасить клетки шахматной доски $4 \times 4$ в чёрный и белый цвета, две раскраски считаются одинаковыми: \begin{enumerate}
            \item если одну можно получить из другой поворотом;
            \item если одна получается из другой под действием элемента группы $D_4$.
    \end{enumerate}

    \item Правильный шестиугольник разбит на 6 равносторонних треугольников. Сколькими способами можно раскрасить эти треугольники в 3 цвета, если расраски совпадающие, при повороте на угол $60^\circ$, считаются одинаковыми?

    \item Молекула имеет форму правильного тетраэдра, в каждой вершине которой может находиться один из атомов: $H$ (водород), $Cl$ (хлор), $Br$ (бром). Сколько существует различных молекул, если молекулы, совпадающие при вращении, считаются одинаковыми, но при отражения (зеркальные изомеры) считаются разными?
\end{enumerate}
