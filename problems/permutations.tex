\subsection*{Задачи}
\begin{enumerate}
    \item Возьмите какое-нибудь четырёхбуквенное слово, скажем, прошлое слово \textsf{УШКА}.
        Покажите, что все варианты (\emph{А сколько, кстати, их?}) тоже разбиваются на две группы,
        и обмен двух букв местами переводит нас из одной группы в другую.
    \item Вова сказал своей подруге, что подарит ей доширак,
        если она в слове \textsf{КОМАНДА} сделает семь попарных обменов и получит исходное слово.
        В чём просчитался Вова?
    \item Найти цикловой тип, порядок и четность перестановки
        \[
            \sigma = \begin{pmatrix}
                1 & 2 & 3 & 4 & 5 & 6 & 7 & 8 & 9 & 10 & 11 \\
                5 & 9 & 1 & 3 & 2 & 11 & 10 & 8 & 4 & 7 & 6
            \end{pmatrix}.
        \]
    \item Найдите все перестановки трехэлементного множества.
    \item Докажите, что любая перестановка имеет обратную.
    \item Найдите обратную перестановку для: 

        \begin{enumerate*}[label=(\alph*), itemjoin=;\hspace{5ex}]
            \item \(\begin{pmatrix}
                    1 & 2 & 3 & 4 \\
                    2 & 1 & 4 & 3
                \end{pmatrix}\)
            \item \(\begin{pmatrix}
                    1 & 2 & 3 & 4 & 5 \\
                    2 & 1 & 4 & 5 & 3
                \end{pmatrix}\).
        \end{enumerate*} 
    \item Верно ли, что композиция двух циклов длины 2 является перестановкой порядка 1 или 2?
    \item Пусть дана перестановка в виде композиции циклов  \[
            \sigma = | 1, 4, 7 \rangle |2, 5 \rangle
        .\]  
        Напишите ее ''матричный вид``, ее порядок и обратную ей.
    \item Пусть даны две перестановки \[
            \sigma = | 1, 4, 2 \rangle 
            , \quad
            \tau = | 1, 3 \rangle | 2, 5 \rangle
        .\]
        Найдите композиции $\tau \circ \sigma$ и $\sigma \circ \tau$, четность и порядок этих композиций.
    \item Пусть даны две перестановки \[
            \sigma = | 1, 8, 5, 2 \rangle | 3, 7 \rangle
            , \quad
            \tau = | 1, 4\rangle |2,  3, 6 \rangle | 5, 8 \rangle
            .
        \]
        Найдите композиции $\tau \circ \sigma$ и $\sigma \circ \tau$, четность и порядок этих композиций.

\end{enumerate}
