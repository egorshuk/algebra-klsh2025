\subsection*{Задачи}
\begin{enumerate}
    \item Возьмите какое-нибудь четырёхбуквенное слово, скажем, прошлое слово \textsf{УШКА}.
        Покажите, что все варианты (\emph{А сколько, кстати, их?}) тоже разбиваются на две группы,
        и обмен двух букв местами переводит нас из одной группы в другую.
    \item Вова сказала своей подруге, что подарит ей доширак,
        если она в слове \textsf{КОМАНДА} сделает семь попарных обменов и получит исходное слово.
        В чём просчитался Вова?
    % Сделай задачу на перестановку из 9 элементов, найти ее цикловой тип, порядок и четность. 
    \item Найти цикловой тип, порядок и четность перестановки
        \[
            \sigma = \begin{pmatrix}
                1 & 2 & 3 & 4 & 5 & 6 & 7 & 8 & 9 \\
                2 & 3 & 1 & 5 & 4 & 7 & 9 & 8 & 6
            \end{pmatrix}.
        \]
    \item Найдите все перестановки трехэлементного множества.
    \item Докажите, что любая перестановка имеет обратную.
    \item Найдите обратную перестановку для: 

        \begin{enumerate*}[label=(\alph*), itemjoin=;\hspace{5ex}]
            \item \(\begin{pmatrix}
                1 & 2 & 3 & 4 \\
                2 & 1 & 4 & 3
            \end{pmatrix}\)
            \item \(\begin{pmatrix}
                1 & 2 & 3 & 4 & 5 \\
                2 & 1 & 4 & 5 & 3
            \end{pmatrix}\).
        \end{enumerate*} 
    \item Верно ли, что композиция двух циклов длины 2 является перестановкой порядка 1 или 2?
\end{enumerate}
