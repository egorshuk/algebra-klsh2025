\documentclass[10pt]{article}
\usepackage{../be-my-concrete, ../extra-algebra, ../print-geometry}
\setcounter{MaxMatrixCols}{15}

\namenumberpage{Перестановки}{1}

\begin{document}

\subsection*{Словарик}
\begin{bullets}
    \item \emph{Перестановка} -- это отображение, которое переставляет элементы множества. У одного такого отображения есть несколько записей: \[\underbrace{\begin{pmatrix}1 & 2 & 3 & 4 & 5 \\ 2 & 3 & 1 & 5 & 4\end{pmatrix}}_{\text{Матричная запись}}\qquad\text{и}\qquad \underbrace{|1 3 2 \rangle | 4 5 \rangle}_{\text{Композиция циклов}}\]

    \item \emph{Цикловым типом} перестановки называется соответсвующая диаграмма Юнга (квадратики в рядах) или набор чисел. Например, у прошлой перестановки цикловой тип $(3, 2)$. А диаграмма Юнга:
\end{bullets}
        \begin{figure}[ht]
            \centering
            \begin{asy}
                size(1.5cm);
                defaultpen(fontsize(10));

                draw(unitsquare);
                draw(shift(1, 0)*unitsquare);
                for (int i = 0; i < 3; ++i) {
                    draw(shift(i, 1)*unitsquare);
                }


                label("1", (0.5, 1.5));
                label("3", (1.5, 1.5));
                label("2", (2.5, 1.5));
                label("4", (0.5, 0.5));
                label("5", (1.5, 0.5));
            \end{asy}
            \hspace{1cm} или просто \hspace{1cm}
            \begin{asy}
                size(1.5cm);
                defaultpen(fontsize(10));

                draw(unitsquare);
                draw(shift(1, 0)*unitsquare);
                for (int i = 0; i < 3; ++i) {
                    draw(shift(i, 1)*unitsquare);
                
    \end{asy}
\end{figure}

\begin{bullets}
\item \emph{Порядок перестановки} обозначает минимальное число раз, которое нужно применить перестановку, чтобы получить изначальную расстановку (т.е. тривиальную перестановку). Например, у перестановки выше порядок равен 6.

\item Перестановка называется \emph{чётной}, если в ней присутствует чётное число циклов чётной длины.

\item \emph{Композицией} двух перестановок $\alpha$ и $\beta$ называется их последоветельное применение. И обозначается $\beta \circ \alpha$ -- тут вначале применяется перестановка $\alpha$, потом $\beta$.

\item \emph{Обратной перестановкой $\alpha^{-1}$} называется такая переставнока, что композиция $\alpha^{-1} \circ \alpha$ тождественна, т.е. не меняет расстановки.

\end{bullets}

\subsection*{Задачи}
\begin{enumerate}
    \item Возьмите какое-нибудь четырёхбуквенное слово, скажем, прошлое слово \textsf{УШКА}.
        Покажите, что все варианты (\emph{А сколько, кстати, их?}) тоже разбиваются на две группы,
        и обмен двух букв местами переводит нас из одной группы в другую.
    \item Вова сказал своей подруге, что подарит ей доширак,
        если она в слове \textsf{КОМАНДА} сделает семь попарных обменов и получит исходное слово.
        В чём просчитался Вова?
    \item Найти цикловой тип, порядок и четность перестановки
        \[
            \sigma = \begin{pmatrix}
                1 & 2 & 3 & 4 & 5 & 6 & 7 & 8 & 9 & 10 & 11 \\
                5 & 9 & 1 & 3 & 2 & 11 & 10 & 8 & 4 & 7 & 6
            \end{pmatrix}.
        \]
    \item Найдите все перестановки трехэлементного множества.
    \item Сколько существует перестановок слова \textsf{РЫБА}, состоящих ровно
        из двух циклов? Найдите эти слова.
    \item Докажите, что любая перестановка имеет обратную.
    \item Найдите обратную перестановку для: 

        \begin{enumerate*}[(a), itemjoin=;\qquad]
            \item \(\begin{pmatrix}
                    1 & 2 & 3 & 4 \\
                    2 & 1 & 4 & 3
                \end{pmatrix}\)
            \item \(\begin{pmatrix}
                    1 & 2 & 3 & 4 & 5 \\
                    2 & 1 & 4 & 5 & 3
                \end{pmatrix}\).
        \end{enumerate*} 
    \item Верно ли, что композиция двух циклов длины 2 является перестановкой порядка 1 или 2?
    \item Пусть дана перестановка в виде композиции циклов  \[
            \sigma = | 1, 4, 7 \rangle |2, 5 \rangle
        .\]  
        Напишите ее ``матричный вид'', ее порядок и обратную ей.
    \item Пусть даны две перестановки \[
            \sigma = | 1, 4, 2 \rangle 
            , \quad
            \tau = | 1, 3 \rangle | 2, 5 \rangle
        .\]
        Найдите композиции $\tau \circ \sigma$ и $\sigma \circ \tau$, четность и порядок этих композиций.
        А также их ``матричный вид''.
    \item Пусть даны две перестановки \[
            \sigma = | 1, 8, 5, 2 \rangle | 3, 7 \rangle
            , \quad
            \tau = | 1, 4\rangle |2,  3, 6 \rangle | 5, 8 \rangle
            .
        \]
        Найдите композиции $\tau \circ \sigma$ и $\sigma \circ \tau$, четность и порядок этих композиций.

\end{enumerate}


\end{document}
