\documentclass[10pt]{article}
\usepackage{../be-my-concrete, ../extra-algebra, ../print-geometry}

\namenumberpage{Отображения и множества}{0}

\begin{document}

\subsection*{Словарик}
\begin{bullets}
    \item Для заданных множеств $X, Y$ их \emph{объединение} $X \cup Y$ состоит из всех элементов, принадлежащих хотя бы одному из множеств $X, Y$; \emph{пересечение} $X \cap Y$ состоит из всех элементов, принадлежащих одновременно каждому из множеств $X, Y$; \emph{разность} $X \setminus Y$ состоит из всех элементов множества $X$, которые не содержатся в $Y$.
    \item Множество $X$, явбляющееся объединением двух непересекающихся множеств $Y$ и $Z$, называется \emph{дизъюнктным объединением} $Y \sqcup Z$.
    \item Множество $X \times Y$, элементами которого являются всевозможные пары $(x, y)$ с $x \in X,\; y \in Y$, называется \emph{декартовым произведением} множеств $X$ и $Y$.
    \item Множество всех таких точек $x \in X$, образ которых равен заданной точке $y \in Y$, называется \emph{полным прообразом} точки $y$ или \emph{слоем} отображения $f$ над $y$ и обозначается \(f^{-1}(y) \defeq \{x \in X \mid f(x) = y \} .\)
    \item Множетсво $y \in Y$, имеющих непустой прообраз, называется образом отображения $f$ и обозначается \( f(X) = \im (f) \defeq \{y \in Y \mid \exists x \in X \colon f(x) = y \}. \)
    \item Отображение $f \colon X \to Y$ называется \emph{наложением} (а также \emph{сюръекцией} или \emph{эпиморфизмом}), если $\im(f) = Y$. Мы будем отображать сюръективные отображения стрелками $f \colon X \twoheadrightarrow Y$. 
    \item Отображение $f \colon X \to Y$ называется \emph{вложение} (а также \emph{инъекцией} или \emph{мономорфизмом}), если $f(x_1) \neq f(x_2)$ при $x_1 \neq x_2$. Мы будем отображать сюръективные отображения стрелками $f \colon X \hookrightarrow Y$. 
    \item Отображение $f \colon X \to Y$, которое является и вложением, и наложемнием называется \emph{взаимно однозначным} (а также \emph{биекцией} или \emph{изоморфизмом}). Мы будем обозначать биекцию стрелками $X \simrightarrow Y$.
    \item Назовём \emph{отношением} на множестве $X$ любое подмножество \(R \subset X\times X = \{(a, b)\mid a, b \in X\}.\)
    \item Отношение $R \subset X \times X$ называется \emph{эквивалетностью}, если оно обладает тремя свойствами: 
    \begin{enumerate*}[(i)]
    \item рефлексивность: $a \sim a, \quad \forall a \in X$;
    \item транзитивность: для любых $a, b, c \in X$ из $a \sim b$ и $b \sim c $ вытекает $a \sim c$;
    \item симметричность: $a \sim b \iff b \sim a, \quad \forall a, b \in X$. \end{enumerate*}

\end{bullets}

\subsection*{Задачи}
\begin{enumerate}
    \item Перечислите все отображения $\{0, 1, 2\} \to \{0, 1\}$ и $\{0, 1\} \to \{0, 1, 2\}$. Сколько среди них сюръекций и инъекций?

\end{enumerate}


\end{document}
