\documentclass[10pt]{article}
\usepackage{../be-my-concrete, ../extra-algebra, ../print-geometry}

\namenumberpage{Алгебра--0,5}{Подсчёт орбит}{5}

\begin{document}
\subsection*{Словарик}
\begin{bullets}
    \item Классом эквивалентности отношения $x \sim y \iff gx = y$ называется \emph{орбитой} точки $x$ под действием $G$ и обозначается 
    \(Gx \defeq \{gx \mid g \in G\}.\)

    Множеством всех орбит называется \emph{фактором} множества $X$ под действием $G$ и обозначается $X/G$.

    \item Слоем отображения $\ev_x \colon G \twoheadrightarrow Gx$ над самой точкой $x$ называется \emph{стабилизатором} точки $x \in X$ и обозначается \( \Stab(x) \defeq \{g \in G \mid gx = x \}.\)
    
    \item Длина орбиты произвольной точки $x \in X$ при действии на неё конечной группы преобразований $G$ равна $|Gx| = \rfrac{|G|}{|\Stab(x)}.$ В частности, длины всех орбит и порядки стабилизаторов всех точек являются делителем порядка группы.

    \item \emph{Формула Пойа-Бернсайда.} Пусть конечная группа $G$ действует на конечном множестве $X$. Для каждого $g \in G$ обозначим, через $$X^g = \{x \in X \mid gx = x \} = \{ x \in X \mid g \in \Stab(x)\}$$ множество неподвижных точек  преобразования $g$. Тогда верна следующая формула \[
            |X/G| = \frac{1}{|G|} \cdot \sum_{g \in G}|X^g|.
        \]
\end{bullets}
\subsection*{Задачки} 
\begin{enumerate}
    \item Сколькими разными способами можно раскрасить грани тетраэдра в $n$ цветов, где две раскраски считаются \emph{одинаковыми}, если одну можно получить поворотов другой.
    \item Сколькими различными способами можно составить ожерелье из $n$ цветов, в котором будет \begin{enumerate}
        \item 5 бусин;
        \item 6 бусин;
        \item 7 бусин.
    \end{enumerate}

    \item Флаг некоторой страны состоит из трёх горизонтальных полос. Сколькими способами можно раскрасить его в $n$ цветов. Если флаги, отличающиеся перестановкой полос, считаются одинаковыми.

    \item Сколькими разными способами можно раскрасить рёбра куба в $n$ цветов, где две раскраски считаются \emph{одинаковыми}, если одну можно получить поворотов другой.

    \item Сколькими способами можно раскрасить клетки шахматной доски $4 \times 4$ в чёрный и белый цвета, две раскраски считаются одинаковыми: \begin{enumerate}
            \item если одну можно получить из другой поворотом;
            \item если одна получается из другой под действием элемента группы $D_4$.
    \end{enumerate}

    \item Правильный шестиугольник разбит на 6 равносторонних треугольников. Сколькими способами можно раскрасить эти треугольники в 3 цвета, если расраски совпадающие, при повороте на угол $60^\circ$, считаются одинаковыми?

    \item Молекула имеет форму правильного тетраэдра, в каждой вершине которой может находиться один из атомов: $H$ (водород), $Cl$ (хлор), $Br$ (бром). Сколько существует различных молекул, если молекулы, совпадающие при вращении, считаются одинаковыми, но при отражения (зеркальные изомеры) считаются разными?
\end{enumerate}

\vfill
\begin{figure}[ht]
    \centering
    \includegraphics[width=0.7\linewidth]{../images/photo_2025-07-25_17-51-18.jpg}
\end{figure}
\vfill
\end{document}
