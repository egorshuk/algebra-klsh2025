\documentclass[10pt]{article}
\usepackage{../be-my-concrete, ../extra-algebra, ../print-geometry}

\namenumberpage{Группы}{2}

\begin{document}
\subsection*{Словарик}
\begin{bullets}
    \item \emph{Группа} -- это множество $G$ с операцией $\star$, которое обладает следующими свойствами: 
    \begin{conditions}
        \item \textit{замкнутость}: $\forall a, b \in G: a \star b \in G$;
        \item \textit{ассоциативность}: $\forall a, b, c \in G: (a \star b) \star c = a \star (b \star c)$;
        \item \textit{наличие нейтрального элемента}: $\exists e \in G: \forall a \in G: e \star a = a$;
        \item \textit{наличие обратного элемента}: $\forall a \in G: \exists a^{-1} \in G: a \star a^{-1} = e$.
    \end{conditions}

\item Для группы также существует обозначение: $(G, \star)$, если операция понятна, то она обозначается $G$.
Если группа $G$ конечна, то ее \emph{порядок} $|G|$ -- это количество элементов в ней.
\emph{Порядком элемента} $a$ же, аналогично с перестановкой, называется такое минимальное число $d$, что $a^d = e$.

\item Множество $H \subset G$ называется \emph{подгруппой} группы $(G, \star)$. Если для нее выполняются аксиомы группы (i)-(iv), то есть \begin{inumerate}[(i)]
        \item $\forall a, b \in H: a \star b \in H$
        \item $\forall a, b, c \in H: (a \star b) \star c = a \star (b \star c)$
        \item $\exists e \in H: \forall a \in H: e \star a = a$
        \item $\forall a \in H: \exists a^{-1} \in H: a \star a^{-1} = e$. 
\end{inumerate}
Тут важно, что вместо группы $G$ написана группа $H$.

\item Группа называется \emph{абелевой (коммутативной)}, если операция $\star$ коммутативна, то есть $\forall a, b \in G: a \star b = b \star a$.

\item \emph{Таблицой Кэли} называется таблица, в которой записаны все элементы группы и их композиции. 

\item Группы называются \emph{изоморфными}, если между ними существует взаимно однозначное соответствие, сохраняющее операцию. То есть, если $\varphi: (G, \star) \to (H, *)$ -- изоморфизм, то $\forall a, b \in G: \varphi(a \star b) = \varphi(a) * \varphi(b)$. Только у изоморфных группы изоморфны таблицы Кэли. Неформально говоря, изоморфные группы -- это ``одни и те же'' группы.
\end{bullets}
\subsection*{Задачки}
\begin{enumerate}
    \item Докажите, что $A_n$ -- подгруппа группы $S_n$.
    \item Верно ли что любая подгруппа абелевой группы является абелевой?
        Если да, то почему? Если нет, то приведите контрпример.
    \item Докажите, что все элементы порядка $2$ и единица в группе $S_n$ образуют подгруппу. 
    \item Пусть $H = (\{-1, 1\}, \cdot)$ в $(\R^*, \cdot).$\footnote{
        Здесь ``звёздочка'' обозначает то, что нет нуля.}
        Является ли $H$ подгруппой?
        Является ли $H$ абелевой группой?
    \item Найдите все подгруппы группы $S_3$. Укажите их порядок. 
        Какие из них абелевы?
    \item Найдите все подгруппы группы $\Z/(6)$. Укажите их порядок. 
        Какие из них абелевы?
    \item Летнешкольников заставили выложить плац правильной шестиугольной плиткой. 
        Сколько существует симметрий такого замощения плиткой? Образуют ли они группу?
        Если да, то какой у нее порядок?
    \item Пусть $H$ -- множество всех перестановок из $S_3$, которые оставляют
        тройку на месте. Является ли $H$ подгруппой группы $S_3$?
        Если да, то какой у нее порядок и является ли она абелевой?
    \item Множество $G = \{2^n \mid n \in \Z\}$ с операцией умножения является ли группой?
        Если да, то является ли она абелевой? Какие в ней подгруппы?
\end{enumerate}

\end{document}

