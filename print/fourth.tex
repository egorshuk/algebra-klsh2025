\documentclass[10pt]{article}
\usepackage{../be-my-concrete, ../extra-algebra, ../print-geometry}

\namenumberpage{Алгебра--0,5}{Действие группы}{4}

\begin{document}
\subsection*{Словарик}
\begin{bullets}
    \item Отображение $\varphi\colon G_1 \to G_2$ называется \emph{гомоморфизмом}, если переводит композицию в композицию, иначе говоря для любых $a,b \in G_1$ в группе $G_2$ выполняется соотношене \(\varphi(ab) = \varphi(a)\varphi(b)\).

    \emph{Эпиморфизмом} называют сюръективный гомоморфизм, \emph{мономорфизмом} называют инъективный гомоморфизм, а \emph{изоморфизмом}, на самом деле, называют биективный гомоморфизм.

    \item Множество всех значений гомоморфизма $\varphi \colon G_1 \to G_2$ называется его \emph{образом} и обозначается $\im \varphi$ или $\varphi (G_1) = \{\varphi(g_1)\mid g_1 \in G_1\}.$

    \item Подмножество $f^{-1}(y) \subset X$ называется \emph{слоем отображения} $f \colon X \to Y$ над точкой $y \in Y$.

    \item Полный прообраз единицы $e_2 \in G_2$ при гомоморфизме групп $\varphi \colon G_1 \to G_2$ называется \emph{ядром} гомоморфизма $\varphi$, обозначается \(\ker \varphi \defeq \{g_1 \in G_1 \mid \varphi(g_1) = e_2 \}.\)

    \item Гомоморфизм $\varphi\colon G \to \Aut(X)$ называется \emph{Действием} группы $G$ на множестве $X$ или \emph{представлением} группа $G$ автоморфизмами множества $X$. 

    \item Классом эквивалентности отношения $x \sim y \iff gx = y$ называется \emph{орбитой} точки $x$ под действием $G$ и обозначается 
    \(Gx \defeq \{gx \mid g \in G\}.\)

    Множеством всех орбит называется \emph{фактором} множества $X$ под действием $G$ и обозначается $X/G$.

    \item Слоем отображения $\ev_x \colon G \twoheadrightarrow Gx$ над самой точкой $x$ называется \emph{стабилизатором} точки $x \in X$ и обозначается \( \Stab(x) \defeq \{g \in G \mid gx = x \}.\)
    
    \item Длина орбиты произвольной точки $x \in X$ при действии на неё конечной группы преобразований $G$ равна $|Gx| = \rfrac{|G|}{|\Stab(x)}.$ В частности, длины всех орбит и порядки стабилизаторов всех точек являются делителем порядка группы.


\end{bullets}
\subsection*{Задачки}


\end{document}
