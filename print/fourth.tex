\documentclass[10pt]{article}
\usepackage{../be-my-concrete, ../extra-algebra, ../print-geometry}

\namenumberpage{Алгебра--0,5}{Действие группы}{4}

\begin{document}
\subsection*{Словарик}
\begin{bullets}
    \item Отображение $\varphi\colon G_1 \to G_2$ называется \emph{гомоморфизмом}, если переводит композицию в композицию, иначе говоря для любых $a,b \in G_1$ в группе $G_2$ выполняется соотношене \(\varphi(ab) = \varphi(a)\varphi(b)\).

    \emph{Эпиморфизмом} называют сюръективный гомоморфизм, \emph{мономорфизмом} называют инъективный гомоморфизм, а \emph{изоморфизмом}, на самом деле, называют биективный гомоморфизм.

    \item Множество всех значений гомоморфизма $\varphi \colon G_1 \to G_2$ называется его \emph{образом} и обозначается $\im \varphi$ или $\varphi (G_1) = \{\varphi(g_1)\mid g_1 \in G_1\}.$

    \item Подмножество $f^{-1}(y) \subset X$ называется \emph{слоем отображения} $f \colon X \to Y$ над точкой $y \in Y$.

    \item Полный прообраз единицы $e_2 \in G_2$ при гомоморфизме групп $\varphi \colon G_1 \to G_2$ называется \emph{ядром} гомоморфизма $\varphi$, обозначается \(\ker \varphi \defeq \{g_1 \in G_1 \mid \varphi(g_1) = e_2 \}.\)

    \item Гомоморфизм $\varphi\colon G \to \Aut(X)$ называется \emph{Действием} группы $G$ на множестве $X$ или \emph{представлением} группа $G$ автоморфизмами множества $X$. 

    \item Классом эквивалентности отношения $x \sim y \iff gx = y$ называется \emph{орбитой} точки $x$ под действием $G$ и обозначается 
    \(Gx \defeq \{gx \mid g \in G\}.\)

    Множеством всех орбит называется \emph{фактором} множества $X$ под действием $G$ и обозначается $X/G$.

    \item Слоем отображения $\ev_x \colon G \twoheadrightarrow Gx$ над самой точкой $x$ называется \emph{стабилизатором} точки $x \in X$ и обозначается \( \Stab(x) \defeq \{g \in G \mid gx = x \}.\)
    
    \item Длина орбиты произвольной точки $x \in X$ при действии на неё конечной группы преобразований $G$ равна $|Gx| = \rfrac{|G|}{|\Stab(x)}.$ В частности, длины всех орбит и порядки стабилизаторов всех точек являются делителем порядка группы.


\end{bullets}
\subsection*{Задачки}
\begin{enumerate}
    \item Найдите длину орбиты и стабилизатор каждой точки тетраэдра и додекаэдра под действием собственной группы этого тела. 
    \item Пусть $\varphi \colon \Z \to \Z/(6)$ гомоморфизм, который $n \mapsto n \pmod 6.$
            Является ли он изоморфизмом? Найдите $\ker \varphi$ и опишите, 
            чем являются элементы $\im \varphi$?
    \item Пусть $H = \{e, |1 2\rangle \}$ подгруппа $S_3$. Постройте гомоморфизм
         $\psi \colon S_3 \to H$, такой что четные перестановки он переводит в  $e$,
         а нечетные в $|1 2 \rangle$. 
         \begin{enumerate}
             \item Проверьте, что гомоморфизм $\psi$ сохраняет композицию.
             \item Найдите $\ker \psi$.
             \item Проверьте, что $S_3/\ker\psi \cong H$.
         \end{enumerate}
     \item Группа $D_4$ действует на множестве вершин квадрата $\{1, 2, 3, 4\}$.
         \begin{enumerate}
             \item Сколько орбит у этого действия?
             \item Найдите стабилизатор вершины 1. Какой группе он изоморфен?
             \item Проверьте, что \(
             |D_4| = |\Stab(1)| \cdot |\Orb(1)|
             .\)
         \end{enumerate}
    \item Группа $S_3$ действует на многочлене $P(x_1, x_2, x_3) = x_1^2 + x_2x_3$,
        переставляя переменные. Найдите орбиты этого действия. Какой стабилизатор у $P$?
    \item Пусть задан гомоморфизм $\mu \colon \Z/(12) \to \Z/(12)$, который $z \mapsto 3z$.
        \begin{enumerate}
            \item Найдите $\ker \mu$.
            \item Постройте таблицу Кэли для $\im \mu$.
            \item Изоморфна ли $\im \mu$ какой-то известной группе?
        \end{enumerate}
    \item Группа $G = \Z / (4)$ действует на множестве $X = \{1, 2, 3, 4\}$ 
        циклическими сдвигами. Запишите соответствующий гомоморфизм $\varphi \colon G \to S_4$.
        Чему равно $\ker \varphi$?
    \item Группа $\Z / (6)$ действует на множестве $X = \{A, B, C\}$ по правилу
         $$k \cdot A = A, \quad k \cdot B = C, \quad k \cdot C = B, \qquad \forall k \in \Z / (6).$$
         Найдите орбиты и стабилизаторы элементов $X$.
    \item Пусть задан гомоморфизм групп $\varphi \colon G \to M$. 
        Докажите, что для любого элемента $h \in \ker \varphi$ и произвольного $g \in G$
        выполняется $ghg^{-1} \in \ker \varphi$.\footnote{
        Этим доказательством вы покажете, что ядра гомоморфизмов являются \emph{нормальными} подгруппами.}
    \item Постройте группу $A_n$ с помощью какого-то гомоморфизма $S_n \to S_n.$
\end{enumerate}

\end{document}
