\documentclass[10pt]{article}
\usepackage{../be-my-concrete, ../extra-algebra, ../print-geometry}

\namenumberpage{Орбита и стабилизатор. Собственные группы фигур}{3}

\begin{document}
\subsection*{Словарик}
% \emph{Гомоморфизмом групп} называется отображение $\varphi: (G, \star) \to (H, *)$ между группами $G$ и $H$, такое что $\varphi(g_1 \star g_2) = \varphi(g_1) * \varphi(g_2)$ для всех $g_1, g_2 \in G$. То есть гомоморфизм уважает операцию.

% \emph{Образом гомоморфизма} $\psi\colon G \to H$ называется подмножество в $H$ и обозначается $\im \psi$. Оно состоит из таких элементов $h$, что существует элемент $g \in G$, что $\psi(g) = h$. То есть это все элементы, которые можно получить из элементов $G$ с помощью гомоморфизма $\psi$. Поэтому такое множество еще обозначется $\psi(G)$.

% \emph{Ядром гомоморфизма} $\mu \colon (R, \circ) \to (T, \diamond)$ называется подмножество в $R$ и обозначается $\ker \mu$. Оно состоит из таких элементов $r$, что $\mu(r) = e_T$, где $e_T$ -- нейтральный элемент группы $T$. То есть это все элементы, которые переходят в нейтральный элемент группы $T$.

Группа \emph{действует} на множестве $X$, если задано отображение $\varphi: G \to \mathrm{Sym}\small(X)$, где $\mathrm{Sym}\small(X)$ --- группа всех перестановок множества $X$. То есть каждый элемент группы $g \in G$ переводит элементы из $X$ в элементы из $X$ какой-то перестановкой. 


\subsection*{Задачки}
\begin{enumerate}
    \item Придумайте свою фигуру, например букву <<Ж>> или прямую пятиугольную призму, найдите её полную и собственную группу движений. 
    \item Напишите \emph{таблицу Кэли (умножения)} для групп $D_3, D_4, D_5$. Изоморфны ли данные группы каким-нибудь вам известным?
    \item Среди несобственных движений тетраэдра есть повороты на $180^\circ$ вокруг осей, проходящих через середины противоположных рёбер. Докажите, что такие движение образуют группу, изоморфную $V_4$.
    \item Докажите или опровергните
        \begin{inumerate}
            \item $D_2 \cong \Z/(2) \times \Z/(2)$   
            \item $D_3 \cong \Z/(2) \times \Z/(3)$
            \item $D_3 \cong A_3 \times \Z/(2)$
            \item $D_6 \cong A_4$
            \item $O_{\text{октаэдр}} \cong O_{\text{куб}}$.
        \end{inumerate}
    \item Проверьте, что полные и собственные группы куба, октаэдра и икосаэдра соотсветнно состоят из $48$ и $24$, $48$ и $24$, $120$ и $60$ движений.
    \item В собственной группе движений тетраэдра найдите для заданной вершины $V$ найдите две величины:  
        \begin{enumerate*}
            \item количество вершин, в которые можно преобразовать вершину $V$ каким-то движением $g$;
            \item количество преобразований, которые оставляют вершину $V$ неподвижной.
        \end{enumerate*}
        Проверьте, что произведение этих чисел дает порядок всей группы.

    \item Классифицируйте все повороты додекаэдра. Сколько их? Сколько в каждом классе?

    \item Докажите, что $SO_{\text{тетраэдр}}$ является подгруппой $SO_{\text{куб}}$.
    \item[9*.] Докажите, что $SO_{\text{куб}} \cong O_{\text{тетраэдр}}$.
\end{enumerate}

\end{document}
