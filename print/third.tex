\documentclass[10pt]{article}
\usepackage{../be-my-concrete, ../extra-algebra, ../print-geometry}

\namenumberpage{Орбита и стабилизатор. Собственные группы фигур}{3}

\begin{document}
\subsection*{Словарик}
% \emph{Гомоморфизмом групп} называется отображение $\varphi: (G, \star) \to (H, *)$ между группами $G$ и $H$, такое что $\varphi(g_1 \star g_2) = \varphi(g_1) * \varphi(g_2)$ для всех $g_1, g_2 \in G$. То есть гомоморфизм уважает операцию.

% \emph{Образом гомоморфизма} $\psi\colon G \to H$ называется подмножество в $H$ и обозначается $\im \psi$. Оно состоит из таких элементов $h$, что существует элемент $g \in G$, что $\psi(g) = h$. То есть это все элементы, которые можно получить из элементов $G$ с помощью гомоморфизма $\psi$. Поэтому такое множество еще обозначется $\psi(G)$.

% \emph{Ядром гомоморфизма} $\mu \colon (R, \circ) \to (T, \diamond)$ называется подмножество в $R$ и обозначается $\ker \mu$. Оно состоит из таких элементов $r$, что $\mu(r) = e_T$, где $e_T$ -- нейтральный элемент группы $T$. То есть это все элементы, которые переходят в нейтральный элемент группы $T$.

Группа \emph{действует} на множестве $X$, если задано отображение $\varphi: G \to \mathrm{Sym}\small(X)$, где $\mathrm{Sym}\small(X)$ --- группа всех перестановок множества $X$. То есть каждый элемент группы $g \in G$ переводит элементы из $X$ в элементы из $X$ какой-то перестановкой. 


\subsection*{Задачки}
\begin{enumerate}
    \item Пусть $\varphi \colon \Z \to \Z/(6)$ гомоморфизм, который $n \mapsto n \pmod 6.$
        \begin{enumerate}
            \item Является ли он изоморфизмом?
            \item Найдите $\ker \varphi.$
            \item Чем являются элементы $\im \varphi$.
        \end{enumerate}
    \item Пусть $H = \{e, |1 2\rangle \}$ подгруппа $S_3$. Постройте гомоморфизм
         $\psi \colon S_3 \to H$, такой что четные перестановки он переводит в  $e.$
         А нечетные в $|1 2 \rangle$. 
         \begin{enumerate}
             \item Проверьте, что гомоморфизм $\psi$ сохраняет композицию.
             \item Найдите $\ker \psi$.
             \item Проверьте, что $S_3/\ker\psi \cong H$.
         \end{enumerate}
     \item Группа $D_4$ действует на множестве вершин квадрата $\{1, 2, 3, 4\}$.
         \begin{enumerate}
             \item Сколько орбит у этого действия?
             \item Найдите стабилизатор вершины 1.
             \item Проверьте, что \(
             |D_4| = |\Stab(1)| \cdot |\Orb(1)|
             .\)
         \end{enumerate}
     \item Группа $S_3$ действует на многочлене $P(x_1, x_2, x_3) = x_1x_2 + x_2x_3$,
         переставляя переменные. Найдите орбиты этого действия. Какой стабилизатор у $P$?
     \item Пусть задан гомоморфизм $\mu \colon \Z/(12) \to \Z/(12)$, который $z \mapsto 3z$.
         \begin{enumerate}
             \item Найдите $\ker \mu$.
             \item Постройте таблицу Кэли для $\im \mu$.
             \item Изоморфна ли $\im \mu$ какой-то известной группе?
         \end{enumerate}

\end{enumerate}

\end{document}
