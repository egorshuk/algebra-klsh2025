\section{Введение в группы}

\subsection{Определение группы}
\begin{definition}
    [Группа]
    Это множество $G$ с операцией $\star$, которое обладает следующими свойствами: 
    \begin{conditions}
        \item \textit{Замкнутость}: $$\forall a, b \in G: a \star b \in G.$$
        \item \textit{Ассоциативность}: $$\forall a, b, c \in G: (a \star b) \star c = a \star (b \star c).$$
        \item \textit{Наличие нейтрального элемента}: $$\exists e \in G: \forall a \in G: e \star a = a.$$
        \item \textit{Наличие обратного элемента}: $$\forall a \in G: \exists a^{-1} \in G: a \star a^{-1} = e.$$
    \end{conditions}
    Для группы также существует обозначение: $(G, \star).$
    Если группа $G$ конечна, то ее порядок $|G|$ -- это количество элементов в ней.
\end{definition}


Существуют различные классификации групп. Например, классификация по типу операции. 
Бывают группы по сложению (аддитивные), то есть с операцией сложения. 
А также бывают группы по умножению (мультипликативные) -- с операцией умножения.

\begin{example}
        $(\Z, +)$ множество целых чисел с операцией сложения.
\end{example}
\begin{example}
    $({\Z}/{(5)}, +)$ множество остатков по модулю 5 с операцией сложения.
\end{example}
\begin{example}
    $(R, \cdot)$ множество действительных чисел с операцией умножения.
\end{example}
\begin{example}
    Как множество -- движения правильной фигуры, а операция тут -- композиция этих движений.
    Например, у нас есть квадрат. Мы можем его поворачивать на 90 градусов,
    а также можем его отражать относительно осей симметрии.
    Тогда у нас получится группа, которая называется $D_4$\footnote{
    Также такую группу можно было назвать $\mathrm{Isom}(\square) $.}, 
    она состоит из 8 элементов (на \cref{fig:group}):

    \begin{multicols}{2}
        \begin{itemize} 
            \item $e$ --- ничего не делать;
            \item $r$ --- поворот на 90 градусов;
            \item $r^2$ --- поворот на 180 градусов;
            \item $r^3$ --- поворот на 270 градусов;
            \item $s_1$ --- отражение относительно оси симметрии по оси $x$;
            \item $s_2$ --- отражение относительно оси симметрии по оси $y$;
            \item $s_3$ --- отражение относительно диагонали, которая идет из левого верхнего угла в правый нижний;
            \item $s_4$ --- отражение относительно диагонали, которая идет из правого верхнего угла в левый нижний.
        \end{itemize}
    \end{multicols}
\end{example}

\begin{figure}[h]
    \centering
    \begin{asy}
        size(5cm);
        draw((0,0)--(1,0)--(1,1)--(0,1)--cycle, linewidth(bp));
        draw((0.5,-0.1)--(0.5,1.1), dashed+blue);
        draw((-0.1,0.5)--(1.1,0.5), dashed+blue);
        draw((-0.1, -0.1)--(1.1, 1.1), dashed+blue);
        draw((1.1, -0.1)--(-0.1, 1.1), dashed+blue);
        label("$s_1$", (0.5,-0.2), blue);
        label("$s_2$", (-0.2, 0.5), blue);
        label("$s_4$", (1.2, 1), blue);
        label("$s_3$", (-0.2, 1), blue);
        draw("$r$", (1.1,0.3){right}..{left}(1.1,0.7), red, arrow=Arrow(TeXHead));
    \end{asy}
    \caption{Группа движений квадрата}
    \label{fig:group}
\end{figure}

\begin{example}[Симметрическая группа]
    До этого мы рассматривали с вами перестановки букв в словах.
    Такие перестановки тоже образуют группу. Она обозначается $S_n$, 
    где $n$ --- количество букв в слове, а называется \emph{симметрической}.
\end{example}

\begin{example}[Группа автоморфизмов множества]\label{ex:AutX}
Все взаимно однозначные отображения множества $X$ в себя (перестановки его элементов) образуют группу. Она обозначается $\Aut X$ и называется \emph{группой автоморфизмов} множества $X$. Если группа $X$ конечна, то $\Aut X \cong S_{|X|}$.
\end{example}

\begin{proposition}
    Правый нейтральный элемент равен левому нейтральному элементу.
\end{proposition}
\begin{proof}
    Пусть $e_l$ --- левый нейтральный элемент, а $e_r$ --- правый нейтральный элемент. Тогда:
    \(
    e_l = e_l \star e_r = e_r.
    \)
\end{proof}
\begin{proposition}
    Если $e$ --- нейтральный элемент группы, то он единственный.
\end{proposition}
\begin{proof}
    Пусть $e_1$ и $e_2$ --- нейтральные элементы группы. Тогда:
    \(
        e_1 = e_1 \star e_2 = e_2.
    \)
\end{proof}
\begin{practice}
    Докажите, что правый обратный элемент равен левому обратному элементу.
\end{practice}
\begin{practice}
    Докажите, что обратный элемент единственный.
\end{practice}
    


\setcounter{footnote}{0}
\subsubsection{Абелевы группы}
\begin{definition}[Абелева группа]
    Группа $G$\footnote{Часто операция опускается и подразумевается, что группа мультипликативна.} 
    называется абелевой, если она коммутативна, то есть: \[
        \forall a, b \in G: ab = ba.
    .\] 
\end{definition}

\setcounter{footnote}{1}
В этом моменте нужно себя спросить: \emph{,,А что, бывает по-другому?!``} И вот оказывается, что бывает.
Для этого, можно рассмотреть один яркий пример. 
\begin{example}
    Пусть у нас есть группа $G$, которая содержит в себе, по крайней мере два элемента: 
    $a = \text{,,надеть носок``}$ и $b = \text{,,надеть ботинок``}$.\footnote{
    Такая группа устроена довольно сложно и в нашем курсе рассматриватьсяне будет. Ее название $F_2$.}
    Тогда одна последовательность действий не приведет к \emph{странным взглядам окружающих,} а другая да.
\end{example}

\begin{practice}
   Какой из этих случаев ,,нормален``, а какой нет? 
\end{practice}

\begin{practice}
    Является ли группа $D_4$ абелевой?
\end{practice}

\begin{practice}
    Приведите свои примеры \emph{абелевых} и \emph{неабелевых} групп. 
\end{practice}

\subsection{Подгруппы}
\begin{definition}
    [Подгруппа]
    Пусть $G$ -- группа. Тогда $H \subset G$ называется подгруппой, если:
    \begin{conditions}
    \item $e \in H$;
    \item $\forall a, b \in H: a \star b \in H$;
    \item $\forall a \in H: a^{-1} \in H$.
    \end{conditions}
\end{definition}

\begin{example}
    Четные целые числа с операцией сложения образуют подгруппу группы $(\Z, +)$.
\end{example}
\begin{example}
    Множество поворотов квадрата образует подгруппу группы $D_4$.
\end{example}
\begin{example}[Знакопеременная группа]
    Множество четных перестановок образует подгруппу группы $S_n$. 
    И такая подгруппа обозначается $A_n$, а называется \emph{знакопеременной.}
\end{example}
\begin{example}[Группа преобразований]
    Классическим примером групп являются \emph{группы преобразований}. Любая подгруппа $G \subset \Aut X$ (в \cref{ex:AutX}) называется \emph{группой преобразований} множества $X$. Как правило мы будем сокращать запись $g(x)$, где $g \in G, x \in X$, до $gx$. Если $X$ наделено дополнительной структурой, то биекции $g \in \Aut X$, сохраняющую эту структуру, образуют подргуппу в группе $\Aut X$, которая обычно называется группой автоморфизмов этой структуры.
\end{example}
\begin{practice}
    Придумайте свои примеры подгрупп.
\end{practice}
\begin{practice}
    Являются ли четные целые числа подгруппой группы $(\Z, \cdot)$?
\end{practice}

\subsubsection{Циклические подгруппы}
\begin{definition}
    [Циклическая подгруппа]
    Наименьшая по включению подгруппа $H\subset G$, содержащая данный элемент $g \in G$, состоит из всевозможных целых степеней $g$, и называется \emph{циклической}, а обозначается $\langle g \rangle$. Она является абелевой.
    
    Наименьшая степепень $n \in \N$, для которого $g^n = e$, называется \emph{порядком элемента} $g$. 
\end{definition}
\begin{remark}
    Порядок элемента --- это не то же самое, что и порядок группы. Например, в группе $\{e, g, g^2, g^3\}$, где $g^4=e$, порядок элемента $g$ равен 4 и совпадает с порядком группы. А порядок элемента $g^2$ равен 2.
\end{remark}
\begin{example}
    Группа $(\Z, +)$ является циклической, так как $G = \langle 1 \rangle$.
\end{example}
\begin{example}
    Группа $(\Z/{(n)}, +)$ является циклической, так как $G = \langle 1 \rangle$.
\end{example}
