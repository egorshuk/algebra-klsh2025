\section{Введение в группы}

\epigraph{Симметрия, как бы широко или узко мы не понимали это слово, есть идея, с помощью которой человек веками пытался объяснить и создать порядок, красоту и совершенство.}{Герман Вейль.}

<<Теория групп>> расшифрует и формализует понятие \emph{симметрии}. Например, когда мы говорим, что у человека симметричное лицо, мы подразумеваем, что мы можем отразить его лицо относительно \emph{оси симметрии}, и оно будет выглядить абсолютно также. Это утверждение о действии. Или, например, снежинка тоже симметрична, но во много большем порядке: мы можем поворачивать ее на $60^\circ$ или на $120^\circ$, можно отражать ее относительно множества осей симметрии или не делать ничего; и все эти преобразования оставят ее вид прежним. Вот такое вот собрание действий и называется группой. 

Вернее, если бы более чеcтным, то группа это более формальное и абстрактное понятие. Например, как и число 3. Обращаясь к нему мы не задумываемся о каком именном числе объектов идет речь: не важно головки сыра это или миллиарды лет. Но с числом работать намного удобнее: складывая два числа, мы совсем-совсем уже забываем о изначальной ,,природе``.
\subsection{Определение группы}
\begin{definition}
    [Группа]
    Это множество $G$ с операцией $\star$, которое обладает следующими свойствами: 
    \begin{conditions}
        \item \textit{Замкнутость}: $$\forall a, b \in G: a \star b \in G.$$
        \item \textit{Ассоциативность}: $$\forall a, b, c \in G: (a \star b) \star c = a \star (b \star c).$$
        \item \textit{Наличие нейтрального элемента}: $$\exists e \in G: \forall a \in G: e \star a = a.$$
        \item \textit{Наличие обратного элемента}: $$\forall a \in G: \exists a^{-1} \in G: a \star a^{-1} = e.$$
    \end{conditions}
    Для группы также существует обозначение: $(G, \star).$
    Если группа $G$ конечна, то ее порядок $|G|$ -- это количество элементов в ней.
\end{definition}

Стоит еще обратить свое внимание на то, насколько общеупотребимое слово выбрано для этого понятия, для казалось бы, такого спицифического объекта симметрий. Это показатель фундаментальности и общности этого понятия.
Группа фиксирует суть симметрии объекта --- не сами движения, а то, как они сочетаются друг с другом. Операция в группе (композиция) кодирует вопрос: 'Если я сделаю преобразование A, а потом преобразование B, какое одно преобразование C даст тот же итоговый эффект?' Именно эта структура (замкнутость, ассоциативность, наличие 'ничего' и 'отмены') делает группы универсальным инструментом для описания симметрий любой природы.

Существуют различные классификации групп. Например, классификация по типу операции. 
Бывают группы по сложению (аддитивные), то есть с операцией сложения. 
А также бывают группы по умножению (мультипликативные) -- с операцией умножения.


\begin{example}
    Группа отражений лица по вертикали называется группой $\Z/(2)$ и состоит ровно из двух элементов: ничего не делать и сделать что-то.
\end{example}
\begin{example}
        $(\Z, +)$ множество целых чисел с операцией сложения.
\end{example}
\begin{example}
    $({\Z}/{(5)}, +)$ множество остатков по модулю 5 с операцией сложения.
\end{example}
\begin{example}
    $(R, \cdot)$ множество действительных чисел с операцией умножения.
\end{example}
\begin{example}
    Как множество -- движения правильной фигуры, а операция тут -- композиция этих движений.
    Например, у нас есть квадрат. Мы можем его поворачивать на 90 градусов,
    а также можем его отражать относительно осей симметрии.
    Тогда у нас получится группа, которая называется $D_4$\footnote{
    Также такую группу можно было назвать $\mathrm{Isom}(\square) $.}, 
    она состоит из 8 элементов (на \cref{fig:group}):

    \begin{multicols}{2}
        \begin{itemize} 
            \item $e$ --- ничего не делать;
            \item $r$ --- поворот на 90 градусов;
            \item $r^2$ --- поворот на 180 градусов;
            \item $r^3$ --- поворот на 270 градусов;
            \item $s_1$ --- отражение относительно оси симметрии по оси $x$;
            \item $s_2$ --- отражение относительно оси симметрии по оси $y$;
            \item $s_3$ --- отражение относительно диагонали, которая идет из левого верхнего угла в правый нижний;
            \item $s_4$ --- отражение относительно диагонали, которая идет из правого верхнего угла в левый нижний.
        \end{itemize}
    \end{multicols}
\end{example}

\begin{figure}[h]
    \centering
    \begin{asy}
        size(5cm);
        draw((0,0)--(1,0)--(1,1)--(0,1)--cycle, linewidth(bp));
        draw((0.5,-0.1)--(0.5,1.1), dashed+blue);
        draw((-0.1,0.5)--(1.1,0.5), dashed+blue);
        draw((-0.1, -0.1)--(1.1, 1.1), dashed+blue);
        draw((1.1, -0.1)--(-0.1, 1.1), dashed+blue);
        label("$s_1$", (0.5,-0.2), blue);
        label("$s_2$", (-0.2, 0.5), blue);
        label("$s_4$", (1.2, 1), blue);
        label("$s_3$", (-0.2, 1), blue);
        draw("$r$", (1.1,0.3){right}..{left}(1.1,0.7), red, arrow=Arrow(TeXHead));
    \end{asy}
    \caption{Группа движений квадрата}
    \label{fig:group}
\end{figure}

\begin{example}[Симметрическая группа]
    До этого мы рассматривали с вами перестановки букв в словах.
    Такие перестановки тоже образуют группу. Она обозначается $S_n$, 
    где $n$ --- количество букв в слове, а называется \emph{симметрической}.
\end{example}

\begin{example}[Группа кубика Рубика]
    Примером группы огромного порядка ($\sim 4,3\cdot 10^{19}$) может быть группа кубика Рубика. Повороты граней кубика Рубика образуют группу. Композиция поворотов --- это наша операция. 'Ничего не делать' --- нейтральный элемент. Обратный элемент --- поворот в обратную сторону или последовательность ходов, отменяющая действие. Эта группа не коммутативна: повернуть верхнюю грань, а потом правую --- не то же самое, что повернуть правую, а потом верхнюю!
\end{example}

\begin{example}[Группа автоморфизмов множества]\label{ex:AutX}
Все взаимно однозначные отображения множества $X$ в себя (перестановки его элементов) образуют группу. Она обозначается $\Aut X$ и называется \emph{группой автоморфизмов} множества $X$. Если группа $X$ конечна, то $\Aut X \cong S_{|X|}$.
\end{example}

Заострим свое внимание на том, насколько важна структура, которую мы хотим сохранить в группах. Допустим у нас есть 6 точек и мы хотим рассмотреть все-все преобразования, которые оставляют их на месте. Мы уже познакомились с таким понятием --- это все  перестановки 6 элементов, а таких 720 штук. 
Если же в точках была дополнительная структура, допустим, они бы образовывали вершины правильного шестиугольника, то движений которые оставляют такой шестиугольник на месте ровно 12 штук. Столько же, сколько и симметрий у снежинки.

\begin{proposition}
    Правый нейтральный элемент равен левому нейтральному элементу.
\end{proposition}
\begin{proof}
    Пусть $e_l$ --- левый нейтральный элемент, а $e_r$ --- правый нейтральный элемент. Тогда:
    \(
    e_l = e_l \star e_r = e_r.
    \)
\end{proof}
\begin{proposition}
    Если $e$ --- нейтральный элемент группы, то он единственный.
\end{proposition}
\begin{proof}
    Пусть $e_1$ и $e_2$ --- нейтральные элементы группы. Тогда:
    \(
        e_1 = e_1 \star e_2 = e_2.
    \)
\end{proof}
\begin{practice}
    Докажите, что правый обратный элемент равен левому обратному элементу.
\end{practice}
\begin{practice}
    Докажите, что обратный элемент единственный.
\end{practice}
    
Да, все эти движения могут показаться интересными, но может возникнуть вопрос зачем это все нужно... Одно из применений групп возникло, когда математики прошлого столкнулись с проблемой решений уравнений различных степеней. Вы уже знакомы с тем, как находить корни квадратного уравнения: для этого есть довольно удобная формула дискриминанта. Возможно, вы знаете, что есть формула для кубического уравнения, а также для уравнения 4-стенени, правда они уже довольно ужасны. В поисках формулы для уравнений пятой степени люди были безуспешны, как оказалось, за этим стоит группа, которая переставляет корни уравнения пятой степени, а именно $S_5$ и её подгруппа $A_5$, вот что-то в их природе такое есть, что не делает разрешимым в радикалах уравнения выше пятой степени. Теория о автоморфизмах корней уравнений называется <<\emph{теорией Эвариста Галуа}>>

Молодой французский математик Эварист Галуа (1811--1832), буквально за ночь перед дуэлью, записал идеи, связавшие разрешимость уравнений в радикалах со структурой групп перестановок их корней. Он ввел само понятие группы (хотя и не использовал этот термин) и показал, почему для уравнений степени 5 и выше общей формулы быть не может. Его работы легли в основу теории Галуа.

Один из самых глубоких результатов, связывающих симметрии и реальный мир --- теорема Эмми Нётер (1918). Грубо говоря, она гласит: Каждой непрерывной симметрии физической системы соответствует закон сохранения. Симметрия относительно сдвигов во времени? Сохранение энергии! Симметрия относительно сдвигов в пространстве? Сохранение импульса! Симметрия относительно поворотов? Сохранение момента импульса! Группы лежат в основе формулировки этих симметрий.

\setcounter{footnote}{0}
\subsubsection{Абелевы группы}
Среди всех групп, абелевы группы, пожалуй самы ,,родные`` и понятные, для большинства людей в мире. Идея о том, что операция коммутативна --- очень приятна, поэтому давайте познакомиися с ними. 

\begin{definition}[Абелева группа]
    Группа $G$\footnote{Часто операция опускается и подразумевается, что группа мультипликативна.} 
    называется абелевой, если она коммутативна, то есть: \[
        \forall a, b \in G: ab = ba.
    .\] 
\end{definition}

\setcounter{footnote}{1}
В этом моменте нужно себя спросить: \emph{,,А что, бывает по-другому?!``} И вот оказывается, что бывает.
Для этого, можно рассмотреть один яркий пример. 

\begin{example}
    В группе $S_3$ если взять, например $|1, 2 \rangle$ и $| 1, 3 \rangle$, То $| 1, 2 \rangle |1, 3 \rangle  = | 1, 3, 2 \rangle$, а если взять в обратном порядке, т.е. $| 1, 3 \rangle | 1, 2 \rangle = | 1, 2, 3 \rangle$. Поэтому группа $S_3$ не абелева!
\end{example}

\begin{example}
    Пусть у нас есть группа $G$, которая содержит в себе, по крайней мере два элемента: 
    $a = \text{,,надеть носок``}$ и $b = \text{,,надеть ботинок``}$.\footnote{
    Такая группа устроена довольно сложно и в нашем курсе рассматриватьсяне будет. Ее название $F_2$.}
    Тогда одна последовательность действий не приведет к \emph{странным взглядам окружающих,} а другая да.
\end{example}


\begin{practice}
   Какой из этих случаев ,,нормален``, а какой нет? 
\end{practice}

\begin{practice}
    Является ли группа $D_4$ абелевой?
\end{practice}

\begin{practice}
    Приведите свои примеры \emph{абелевых} и \emph{неабелевых} групп. 
\end{practice}

\subsection{Подгруппы}
\begin{definition}
    [Подгруппа]
    Пусть $G$ -- группа. Тогда $H \subset G$ называется подгруппой, если:
    \begin{conditions}
    \item $e \in H$;
    \item $\forall a, b \in H: a \star b \in H$;
    \item $\forall a \in H: a^{-1} \in H$.
    \end{conditions}
\end{definition}

\begin{example}
    Четные целые числа с операцией сложения образуют подгруппу группы $(\Z, +)$.
\end{example}
\begin{example}
    Множество поворотов квадрата образует подгруппу группы $D_4$.
\end{example}
\begin{example}[Знакопеременная группа]
    Множество четных перестановок образует подгруппу группы $S_n$. 
    И такая подгруппа обозначается $A_n$, а называется \emph{знакопеременной.}
\end{example}
\begin{example}[Группа преобразований]
    Классическим примером групп являются \emph{группы преобразований}. Любая подгруппа $G \subset \Aut X$ (в \cref{ex:AutX}) называется \emph{группой преобразований} множества $X$. Как правило мы будем сокращать запись $g(x)$, где $g \in G, x \in X$, до $gx$. Если $X$ наделено дополнительной структурой, то биекции $g \in \Aut X$, сохраняющую эту структуру, образуют подргуппу в группе $\Aut X$, которая обычно называется группой автоморфизмов этой структуры.
\end{example}
\begin{practice}
    Придумайте свои примеры подгрупп.
\end{practice}
\begin{practice}
    Являются ли четные целые числа подгруппой группы $(\Z, \cdot)$?
\end{practice}

\subsubsection{Циклические подгруппы}
\begin{definition}
    [Циклическая подгруппа]
    Наименьшая по включению подгруппа $H\subset G$, содержащая данный элемент $g \in G$, состоит из всевозможных целых степеней $g$, и называется \emph{циклической}, а обозначается $\langle g \rangle$. Она является абелевой.
    
    Наименьшая степепень $n \in \N$, для которого $g^n = e$, называется \emph{порядком элемента} $g$. 
\end{definition}
\begin{remark}
    Порядок элемента --- это не то же самое, что и порядок группы. Например, в группе $\{e, g, g^2, g^3\}$, где $g^4=e$, порядок элемента $g$ равен 4 и совпадает с порядком группы. А порядок элемента $g^2$ равен 2.
\end{remark}

Циклическая подгруппа, порожденная элементом $g$, похожа на циферблат часов с $n$ делениями (где $n$ --- порядок $g$). Умножение на $g$ --- это поворот стрелки на одно деление. Умножение на $g^k$ --- поворот на $k$ делений. ,,Обнуление`` ($g^n = e$) происходит, когда стрелка делает полный круг.

\begin{example}
    Группа $(\Z, +)$ является циклической, так как $G = \langle 1 \rangle$.
\end{example}
\begin{example}
    Группа $(\Z/{(n)}, +)$ является циклической, так как $G = \langle 1 \rangle$.
\end{example}

\subsection{О классификации конечных групп}
Вопросом о классификации всех возможных конечных групп, вы наврядли задались, но я постараюсь немного на него ответить. 

Во-первых, мы всегда задаемся вопросом насколько сильно классифицировать группы, будем ли мы считать группу преобразований черного листа А4 отличной от группы преобразований синего листа А4? И конечно, хочется сказать нет, и мы так и сделаем, ответив более формально.

Мы считаем группы с точностью до их изоморфизма. Это означает, что если мы сможем сопоставить взаимно однозначное соотвествием между элементами двух групп, при этом также между тем, как эти элименту умножаются, то такие группы будут одинаковыми.

\begin{example}
    Например, как мы скоро узнаем, группа движений правильного треугольника изоморфна перестановкам из трех элементов. А группа перестановок из 4 элементов изоморфна группе вращений куба (изоморфизм строится по диагоналям.)
\end{example}

Кажется, теперь пора сказать, что этот вопрос оказался чрезвычайно сложным. И был решен лишь в 2004 году. Математики нашли все \emph{простые} группы, ,,переодическую таблицу групп``. Эта таблица состоит из 18 бесконечных семейств групп, а также 26 отдельных <<\emph{спорадических групп}>>. Одним из семейстов являются все простые циклические группы: $\Z/(2), \Z/(3), \Z/(5), \Z/(7), \ldots$, еще одним семейтвом являются четные перестановки $A_5, A_6, A_7, \ldots$. Остальные 16 групп являются \emph{группами Ли}, которые намного сложнее.

Но самыми интересными и непонятными являются те самые 26 спорадических групп, которые так странно выглядят, в таком фундаментальном месте, каФк теория групп. Наибольшей из них является известная \emph{группа Монстра} Джона Конвея. Ее порядок равен \\$808,017,424,794,512,875,886,459,904,961,710,757,005,754,368,000,000,000$. Следующая по порядку, и это совсем не шутка, называется \emph{маленький Монстр (baby monster group)}. На самом деле 20 из 26 относятся к семье ,,монстров``, а остальные 6 называется \emph{изгоями}. Как и прошлые группы, группа Монстра описывает симметрии какого-то объекта, но не как мы привыкли в 2- или 3-мерном пространстве, и даже не 4 --- 196883-мерном пространстве. Один элемент такой группы занимает без сжатия 4 гигабайта памяти в компьютере.

В 70-х годах прошлого столетия математик Джон Маккей перестает заниматься конечным группами и начинает заниматься смежной с теорией групп теорией Галуа. И удивительным образом находит число очень похожее на 196833, выплывшее в совершенно не связаном месте. Число на 1 больше данного возникло в разложении в ряд одной из фондументальнейших функций теории модулярных форм и эллиптических функций. Эту проблему назвали <<Monstorus moonshine>> (гипотеза чудовищного вздора). Позже в 1992 Ричард Боршердс доказал эту связь, кстати, спустя 6 лет он выиграл Филдовсую медаль. Это достижение позволило установить связь между группой Монстра и теорией струн.
