\subsection{Определение группы}

\begin{definition}
    [Группа]
    Это множество $G$ с операцией $\star$, которое обладает следующими свойствами: 
    \begin{conditions}
        \item \textit{Замкнутость}: $$\forall a, b \in G: a \star b \in G.$$
        \item \textit{Ассоциативность}: $$\forall a, b, c \in G: (a \star b) \star c = a \star (b \star c).$$
        \item \textit{Наличие нейтрального элемента}: $$\exists e \in G: \forall a \in G: e \star a = a \star e = a.$$
        \item \textit{Наличие обратного элемента}: $$\forall a \in G: \exists b \in G: a \star b = b \star a = e.$$
    \end{conditions}
    Для группы также существует обозначение: $(G, \star).$
\end{definition}

Существуют различные классификации групп. Например, классификация по типу операции. 
Бывают группы по сложению (аддитивные), то есть с операцией сложения. 
А также бывают группы по умножению (мультипликативные) -- с операцией умножения.

\begin{example}
        $(\Z, +)$ множество целых чисел с операцией сложения.
\end{example}
\begin{example}
    $({\Z}/{(5)}, +)$ множество остатков по модулю 5 с операцией сложения.
\end{example}
\begin{example}
    Как множество -- движения правильной фигуры, а операция тут -- композиция этих движений.
    Например, у нас есть квадрат. Мы можем его поворачивать на 90 градусов,
    а также можем его отражать относительно осей симметрии.
    Тогда у нас получится группа, которая называется $D_4$, она состоит из 8 элементов:

    \begin{multicols}{2}
        \begin{itemize} 
            \item $e$ -- ничего не делать;
            \item $r$ -- поворот на 90 градусов;
            \item $r^2$ -- поворот на 180 градусов;
            \item $r^3$ -- поворот на 270 градусов;
            \item $s_1$ -- отражение относительно оси симметрии по оси $x$;
            \item $s_2$ -- отражение относительно оси симметрии по оси $y$;
            \item $s_3$ -- отражение относительно диагонали, которая идет из левого верхнего угла в правый нижний;
            \item $s_4$ -- отражение относительно диагонали, которая идет из правого верхнего угла в левый нижний.
        \end{itemize}
    \end{multicols}
\end{example}

\begin{figure}[h]
    \centering
    \begin{asy}
        size(5cm);
        draw((0,0)--(1,0)--(1,1)--(0,1)--cycle, linewidth(bp));
        draw((0.5,-0.1)--(0.5,1.1), dashed+blue);
        draw((-0.1,0.5)--(1.1,0.5), dashed+blue);
        draw((-0.1, -0.1)--(1.1, 1.1), dashed+blue);
        draw((1.1, -0.1)--(-0.1, 1.1), dashed+blue);
        label("$s_1$", (0.5,-0.2), blue);
        label("$s_2$", (-0.2, 0.5), blue);
        label("$s_4$", (1.2, 1), blue);
        label("$s_3$", (-0.2, 1), blue);
        draw("$r$", (1.1,0.3){right}..{left}(1.1,0.7), red, arrow=Arrow(TeXHead));
    \end{asy}
    \caption{Группа движений квадрата}
\end{figure}

\begin{example}
    До этого мы рассматривали с вами перестановки букв в словах.
    Такие перестановки тоже образуют группу. Она обозначается $S_n$, 
    где $n$ -- количество букв в слове.
\end{example}


\begin{definition}[Абелева группа]
    Группа $G$\footnote{Часто операция опускается и подразумевается, что группа мультипликативна} 
    называется абелевой, если она коммутативна, то есть: \[
        \forall a, b \in G: ab = ba.
    .\] 
\end{definition}

В этом моменте нужно себя спросить: \emph{``А что, бывает по-другому?!''} И вот оказывается, что бывает.
Для этого, можно рассмотреть один яркий пример. 
\begin{example}
    Пусть у нас есть группа $G$, которая содержит в себе, по крайней мере два элемента: 
    $a = \text{``надеть носок''}$ и $b = \text{``надеть ботинок''}$.\footnote{
    Такая группа устроена довольно сложно и в нашем курсе рассматриваться не будет. Ее название $F_2$.}
    Тогда одна последовательность действий не приведет к \emph{странным взглядам окружающих,} а другая да.
\end{example}

\begin{practice}
   Какой из этих случаев ``нормален'', а какой нет? 
\end{practice}
\begin{practice}
    Приведите свои примеры \emph{абелевых} и \emph{неабелевых} групп. 
\end{practice}
