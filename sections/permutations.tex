\section{Введение в группы}

Алгебра -- наука о структурах, которые описываются с помощью операций и законов. 
Возможно, то что мы будем называть ``алгебра'' -- это не совсем то, что вы привыкли называть ``алгебра''. 
Потому что в школьном курсе алгебры, особенно в старших классах, почему-то изучается анализ, а не сама алгебра.

Первая структура, с которой мы с вами подготовимся -- это группы.
Группы -- это один из основных объектов алгебры. Это самое ``базовое понятие'', 
но оно же и является центральным. 

\subsection{Перестановки}

Самая интерпретируемая группа -- это группа перестановок.
Вероятно, вы уже слышали о том, что такое перестановка, не задумываясь о её групповой структуре.
Для начала, ``нестрого'' разберемся с перестановками.

\begin{example}
    Напишем, какое-нибудь слово, например: \[
        \text{\textsf{УШКА}}\footnote{Это слово осмысленное,
        но в дальнейшем, мы будем называть ``словами'' любые цепочки букв,
    не заботясь о том, являются ли они словами русского языка.}
    \] 
    За один шаг разрешается поменять местами любые две буквы.
    Например, можно поменяв буквы \textsf{А} и \textsf{К}, получить слово \[
        \text{\textsf{УШАК}}
    \] 
\end{example}

\begin{practice}
    Можно ли получить слово \textsf{КАШУ} из слова \textsf{УШКА} за один шаг?
    Если нет, то за какое минимальное число шагов можно это сделать?
\end{practice}

\begin{practice}\label{practice:word}
    Можно ли, начав, со слова \textsf{ТАПОК}, вернуться в исходное слово после 10 шагов?
    После 11 шагов?
\end{practice}

Решив \cref{practice:word}, вы заметили, что за 10 шагов все получилось. А вот за 11 -- никак.
На самом деле это не случайность, и верен более общий факт. 
\begin{proposition}
    Если на каждом шаге разрешено поменять только две буквы, 
    то за нечетное число шагов не получится вернуться в исходное слово.
\end{proposition}

Теперь возьмём другое слово, допустим, \textsf{АДО}. Есть три пары букв, которые можно поменять.
Так что, за один шаг мы можем получить три слова. 
\[
    \text{\textsf{ОДА}} \qquad \text{\textsf{ДАО}} \qquad \text{\textsf{АОД}}
\]
На втором шаге мы должны выбрать одно из этих слов и поменять в нём две буквы. 
Пару для обмена в каждом слове можно выбрать двумя способами, 
а два другие дадут новые слова. 
\begin{align*}
    \text{\textsf{ОДА}} &\to \text{\textsf{ДОА ОАД АДО}} \\
    \text{\textsf{ДАО}} &\to \text{\textsf{ДОА ОАД АДО}}\\
    \text{\textsf{АОД}} &\to \text{\textsf{ДОА ОАД АДО}}
\end{align*}
Видно, что в результате получаются одни и те же три слова.
\[
    \text{\textsf{ДОА}} \qquad \text{\textsf{ОАД}} \qquad \text{\textsf{АДО}}
\] 
\begin{practice}
    Проверьте, что за три шага получается тот же набор слов, что и за 1 шаг.
\end{practice}
Видно, что мы разбили все варианты на две группы по три слова
и на каждом шаге переходим из одной группу в другую: \[
    \text{\textsf{АДО ОАД ДОА}} \leftrightarrow \text{\textsf{ДАО АОД ОДА}}
\] 
А значит, вернуться в исходную группу (в частности, получить слово \textsf{АДО})
можно только за четное число шагов.

\begin{tasks}
    \item Возьмите какое-нибудь четырёхбуквенное слово, скажем, прошлое слово \textsf{УШКА}.
        Покажите, что все варианты (\emph{А сколько, кстати, их?}) тоже разбиваются на две группы,
        и обмен двух букв местами переводит нас из одной группы в другую.
    \item Вова сказала своей подруге, что подарит ей доширак,
        если она в слове \textsf{КОМАНДА} сделает семь попарных обменов и получит исходное слово.
        В чём просчитался Вова?
\end{tasks}

