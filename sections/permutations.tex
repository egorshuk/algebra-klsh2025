\section{Перестановки}

Самая интерпретируемая группа -- это группа перестановок.
Вероятно, вы уже слышали о том, что такое перестановка, не задумываясь о её групповой структуре.
Для начала, ``нестрого'' разберемся с перестановками.

\begin{practice}
    Сколько есть способов переставить $n$ человек в очереди?
\end{practice}

\begin{example}
    Напишем, какое-нибудь слово, например: \[
        \text{\textsf{УШКА}}\footnote{Это слово осмысленное,
        но в дальнейшем, мы будем называть ``словами'' любые цепочки букв,
    не заботясь о том, являются ли они словами русского языка.}
    \] 
    За один шаг разрешается поменять местами любые две буквы.
    Например, можно поменяв буквы \textsf{А} и \textsf{К}, получить слово \[
        \text{\textsf{УШАК}}
    \] 
\end{example}

\begin{practice}
    Можно ли получить слово \textsf{КАШУ} из слова \textsf{УШКА} за один шаг?
    Если нет, то за какое минимальное число шагов можно это сделать?
\end{practice}

\begin{practice}\label{prac:word}
    Можно ли, начав, со слова \textsf{ТАПОК}, вернуться в исходное слово после 10 шагов?
    После 11 шагов?
\end{practice}

В \cref{prac:word}, вы заметили, что за 10 шагов все получилось. А вот за 11 -- никак.
На самом деле это не случайность, и верен более общий факт. 
\begin{proposition}
    Если на каждом шаге разрешено поменять только две буквы, 
    то за нечетное число шагов не получится вернуться в исходное слово.
\end{proposition}

Теперь возьмём другое слово, допустим, \textsf{АДО}. Есть три пары букв, которые можно поменять.
Так что, за один шаг мы можем получить три слова. 
\[
    \text{\textsf{ОДА}} \qquad \text{\textsf{ДАО}} \qquad \text{\textsf{АОД}}
\]
На втором шаге мы должны выбрать одно из этих слов и поменять в нём две буквы. 
Пару для обмена в каждом слове можно выбрать двумя способами, 
а два другие дадут новые слова. 
\begin{align*}
    \text{\textsf{ОДА}} &\to \text{\textsf{ДОА ОАД АДО}} \\
    \text{\textsf{ДАО}} &\to \text{\textsf{ДОА ОАД АДО}}\\
    \text{\textsf{АОД}} &\to \text{\textsf{ДОА ОАД АДО}}
\end{align*}
Видно, что в результате получаются одни и те же три слова.
\[
    \text{\textsf{ДОА}} \qquad \text{\textsf{ОАД}} \qquad \text{\textsf{АДО}}
\] 
\begin{practice}
    Проверьте, что за три шага получается тот же набор слов, что и за 1 шаг.
\end{practice}
Видно, что мы разбили все варианты на две группы по три слова
и на каждом шаге переходим из одной группу в другую: \[
    \text{\textsf{АДО ОАД ДОА}} \leftrightarrow \text{\textsf{ДАО АОД ОДА}}
\] 
А значит, вернуться в исходную группу (в частности, получить слово \textsf{АДО})
можно только за четное число шагов.

\subsection{Нотация перестановок}
\begin{definition}
    [Перестановка]
    Перестановка -- это функция, которая отображает множество букв в себя.
    \[
        \sigma: \{1, 2, \ldots, n\} \to \{1, 2, \ldots, n\}.
    \]
    
    Перестановка $\sigma$ может быть записана в виде\footnote{
    Существуют и другая запись: $(\sigma(1) \; \sigma(2) \; \sigma(3) \; \ldots \; \sigma(n))$.} \[
        \sigma = \begin{pmatrix}
            1 & 2 & 3 & \ldots & n \\
            \sigma(1) & \sigma(2) & \sigma(3) & \ldots & \sigma(n)
        \end{pmatrix}.
    \]
\end{definition}
\begin{example}
    При перестановка $\sigma = \begin{pmatrix}
        1 & 2 & 3 & 4 \\
        2 & 1 & 4 & 3
    \end{pmatrix}$ 
    Первая буква переходит на вторую, вторая -- на первую, третья -- на четвёртую, четвёртая -- на третью.
    Допустим, со словом \textsf{РЫБА} наша перестановка сделает (на \cref{fig:permutation}):
    \[
        \sigma(\text{\textsf{РЫБА}}) = \text{\textsf{ЫРАБ}}.
    \]

    \begin{figure}[ht]
        \centering
        \begin{asy}
            size(7cm);
            defaultpen(fontsize(10));

            pair[] start = {(0,0), (1,0), (2,0), (3,0)};
            string[] startText = {"Р","Ы","Б","А"};

            pair[] end = {(0,-2), (1,-2), (2,-2), (3,-2)};
            string[] endText = {"Ы","Р","А","Б"};

            // Подписи букв
            for(int i = 0; i < 4; ++i) {
                label(startText[i], start[i], N);
                label(endText[i], end[i], S);
            }

            // Кривые "ниточки"
            draw(start[0]..(0.3, -0.5)..end[1], blue+linewidth(1)); // Р -> Ы
            draw(start[1]..(0.6, -0.5)..end[0], blue+linewidth(1)); // Ы -> Р
            draw(start[2]..(2.4, -1.4)..end[3], longdashed+red+linewidth(1)); // Б -> А
            draw(start[3]..(2.8, -1.4)..end[2], longdashed+red+linewidth(1)); // А -> Б

            
        label("$\begin{pmatrix} 2 & 2 & 3 & 4 \\ 2 & 1 & 4 & 3 \end{pmatrix}$", (4.5, -1));
        \end{asy}
        \caption{Перестановка $\sigma$.}
        \label{fig:permutation}
    \end{figure}
\end{example}

\setcounter{footnote}{0}
Применяя одну перестановку за другой, мы можем получить новую перестановку. 
Для этого тоже есть запись. Пусть у нас есть две перестановки $\sigma$ и $\tau$.
Тогда их произведение $\sigma \circ \tau$ -- это перестановка, которая получается из $\tau$,
после чего к ней применяют $\sigma$\footnote{Да! Именно так! Слева-направо!}. 

\begin{example}
    Пусть $\sigma = \begin{pmatrix}
        1 & 2 & 3 & 4 \\
        2 & 1 & 4 & 3
    \end{pmatrix}$ и $\tau = \begin{pmatrix}
        1 & 2 & 3 & 4 \\
        2 & 4 & 3 & 1
    \end{pmatrix}$.
    Тогда их произведение будет равно: 
    \[
        \sigma \circ \tau = \begin{pmatrix}
            1 & 2 & 3 & 4 \\
            \sigma(\tau(1)) & \sigma(\tau(2)) & \sigma(\tau(3)) & \sigma(\tau(4))
        \end{pmatrix} = \begin{pmatrix}
            1 & 2 & 3 & 4 \\
            1 & 3 & 4 & 2
        \end{pmatrix}.
    \]
    
    Давайте рассмотрим, что у нас происходит на примере слова \textsf{КИНО} (на \cref{fig:permutation2}).
    
    \begin{figure}[h]
        \centering
        \begin{asy}
            size(7cm);
            defaultpen(fontsize(10));

            pair[] start = {(0,0), (1,0), (2,0), (3,0)};
            string[] startText = {"К","И","Н","О"};
            pair[] mid = {(0, -2), (1, -2), (2, -2), (3, -2)};
            string[] midText = {"О","К","Н","И"};
            pair[] end = {(0, -4), (1, -4), (2, -4), (3, -4)};
            string[] endText = {"К","О","И","Н"};

            for(int i = 0; i < 4; ++i) {
                label(startText[i], start[i], N);
                label(midText[i], mid[i], N);
                label(endText[i], end[i], S);
            }

            // Кривые "ниточки"
            draw(start[0]{(1, -1)}..(mid[1]+(0, 0.3)), blue+linewidth(1)); // К -> Н
            draw(start[1]{(1, -2)}..(mid[3]+(0, 0.3)), blue+linewidth(1)); // И -> О
            draw(start[2]..(mid[2]+(0, 0.3)), blue+linewidth(1)); // Н -> К
            draw(start[3]{(-6, -1)}..(mid[0]+(0, 0.3)), blue+linewidth(1)); // О -> И

            draw(mid[0]{(0.5, -2)}..end[1], longdashed+red+linewidth(1)); // Н -> О
            draw(mid[1]{(1, -2)}..end[0], longdashed+red+linewidth(1)); // О -> Н
            draw(mid[2]{(3, -2)}..end[3], longdashed+red+linewidth(1)); // К -> И
            draw(mid[3]{(-5, -2)}..end[2], longdashed+red+linewidth(1)); // И -> К

            label(scale(0.9)*"$\begin{pmatrix} 1 & 2 & 3 & 4 \\ 2 & 4 & 3 & 1 \end{pmatrix}$", (4.5, -0.7));
            label(scale(0.9)*"$\begin{pmatrix} 1 & 2 & 3 & 4 \\ 2 & 1 & 4 & 3 \end{pmatrix}$", (4.5, -2.7));
            
        \end{asy}
        \caption{Перестановка $\sigma \circ \tau$.}
        \label{fig:permutation2}
    \end{figure}
\end{example}

\begin{practice}
    Найдите композицию $\tau \circ \sigma$. Проверьте, что это не то же самое, что $\sigma \circ \tau$.
\end{practice}

\begin{definition}
    [Циклическая запись]
    Любую перестановку можно записать в виде произведения циклов.
    Например, перестановка $\sigma$ (на \cref{fig:permutation4}) \[
        \sigma = \begin{pmatrix}
            1 & 2 & 3 & 4 & 5 & 6 \\
            3 & 1 & 5 & 6 & 2 & 4
        \end{pmatrix}.
    \] 
    Записывается в виде: \[
        \sigma = |1 \; 3 \; 5 \; 2\rangle |4 \; 6\rangle.
    \]
    У такой записи есть ``свобода выбора''. Один и тот же цикл можно записать по-разному. 
    Например, \[
        |1 \; 3 \; 5 \; 2\rangle = |3 \; 5 \; 2 \; 1\rangle = |5 \; 2 \; 1 \; 3\rangle = |2 \; 1 \; 3 \; 5\rangle.
    \]

    Мы будем говорить, что у перестановки $\sigma$ цикловой тип $\left( 4,2 \right)$ 
    (в данном случае, это значит, что у нас есть один 4-цикл и один 2-цикл).
    А иногда еще будем рисовать диаграмму Юнга (на \cref{fig:permutation3}), данного циклового типа.

    \begin{figure}[ht]
        \centering
        \begin{asy}
            size(2cm);
            defaultpen(fontsize(10));

            draw(unitsquare);
            draw(shift(1, 0)*unitsquare);
            for (int i = 0; i < 4; ++i) {
                draw(shift(i, 1)*unitsquare);
            }
                

            label("1", (0.5, 1.5));
            label("3", (1.5, 1.5));
            label("5", (2.5, 1.5));
            label("2", (3.5, 1.5));
            label("4", (0.5, 0.5));
            label("6", (1.5, 0.5));
        \end{asy}
        \hspace{1cm} или просто \hspace{1cm}
        \begin{asy}
            size(2cm);
            defaultpen(fontsize(10));

            draw(unitsquare);
            draw(shift(1, 0)*unitsquare);
            for (int i = 0; i < 4; ++i) {
                draw(shift(i, 1)*unitsquare);
            }
        \end{asy}
        \caption{Цикловой тип перестановки $\sigma$.}
        \label{fig:permutation3}
    \end{figure}

\end{definition}
\begin{figure}[ht]
    \centering
    \begin{asy}
        size(7cm);
        defaultpen(fontsize(10));

        pair[] start = {(0,0), (1,0), (2,0), (3,0), (4,0), (5,0)};
        string[] startText = {"1","2","3","4","5","6"};
        pair[] end = {(0,-2), (1,-2), (2,-2), (3,-2), (4,-2), (5,-2)};
        string[] endText = {"3","1","5","6","2","4"};
        // Подписи букв
        for(int i = 0; i < 6; ++i) {
            label(startText[i], start[i], N);
            label(endText[i], end[i], S);
        }

        // Кривые "ниточки"
        draw(start[0]{(1, -2)}..end[2], blue+linewidth(1)); // 1 -> 3
        draw(start[1]{(1.2, -2)}..end[4], blue+linewidth(1)); // 2 -> 5
        draw(start[2]{(1, -2)}..end[1], blue+linewidth(1)); // 3 -> 1
        draw(start[3]{(6, -1)}..end[5], longdashed+red+linewidth(1)); // 4 -> 6
        draw(start[4]{(-3, -1)}..end[0], blue+linewidth(1)); // 5 -> 2
        draw(start[5]{(-3, -2)}..end[3], longdashed+red+linewidth(1)); // 6 -> 4

        label(scale(0.8)*"$\begin{pmatrix} 1 & 2 & 3 & 4 & 5 & 6 \\ 3 & 1 & 5 & 6 & 2 & 4 \end{pmatrix}$", (8, -1));
    \end{asy}
    \caption{Циклическая запись перестановки.}
    \label{fig:permutation4}
\end{figure}

Есть одно важное понятие, которое может таким не показаться.
Возможно, мы не сможем в полном объеме раскрыть его в этом курсе, но что же поделать.
Перед этим, скажем, что \emph{транспозиция} -- это перестановка, которая меняет местами только две буквы.

\begin{definition}
    [Четность перестановки]
    Перестановка называется четной, если она может быть записана в виде произведения четного числа транспозиций.
    Иначе, она называется нечетной.
\end{definition}

\begin{definition}
    [Порядок перестановки]
    Порядок перестановки $\sigma$ -- это наименьшее число $n$, такое что \[
        \sigma^n = \text{id}.
    \]
\end{definition}

\begin{theorem}
    [Порядок перестановки]
    Порядок перестановки $\sigma$ равен наибольшему общему делителю длин всех циклов в её циклической записи.
\end{theorem}
\begin{proof}
    Цикл длины $k_i$ возвращает элементы на место после $k_i$ применений. Поскольку циклы не пересекаются, порядок всей перестановки — минимальное число $k$, при котором $k$ делится на каждое $k_i$.  
    Это и есть наименьшее общее кратное $k_1, k_2, \dots, k_m$.
\end{proof}
