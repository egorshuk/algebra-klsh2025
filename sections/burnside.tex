\subsection{Подсчёт орбит}
Подсчёт числа элементов в факторе $X/G$ (числа орбит) конечного множества $X$ под действием конечной группы $G$ наталкивается на очевидную трудность: посколько длины орбит бывают разными, количество орбит ,,разных типов`` придётся считать по отдельности, по ходу дела уточняя, что такое ,,тип орбиты``. С этим нам поможет следующее утверждение. 
\begin{theorem}\label{th:burnside}
    [Формула Пойа-Бернсайда]
    Пусть конечная группа $G$ действует на конечном множестве $X$. Для каждого $g \in G$ обозначим, через $X^g = \{x \in X \mid gx = x\} = \{x \in X \mid g \in \Stab(x)\}$ множество неподвижных точек преобразования $g$. Тогда верная следующая формула \[
        |X/G| = \frac{1}{|G|}\cdot \sum_{g \in G} |X^g|.
    \]
\end{theorem}
\begin{proof}
    Обозначим через $F \subset G \times X$ таких пар, что $gx = x$. Такое множество можно описать двумя способами: 
    \[
        F = \bigsqcup_{g\in G} X^g = \bigsqcup_{x \in X} \Stab(x).
    \]
    Первое получается рассмотрением проекции $F \twoheadrightarrow G, (g, x) \mapsto g$: слой над точкой $g$ и есть $^g$. Второе получается проекцией $F \twoheadrightarrow X, (g, x)\mapsto x$: слой над точкой $x$ суть $\Stab(x)$. Согласно первому описанию $|F| = \sum_{g \in G} |X^g|$, а по второму $|F| = \sum_{x \in X} |\Stab(x)| = |G| \cdot |X/G|.$
\end{proof}

\begin{example}[Ожерелья]
    Задача ,,ожерелья`` является одной из классических задач теории групп, действия группы на множестве. На её примере рассмотрим применение формулы \emph{Пойа-Бернсайда}. Эта проблема формулируется обычно как-то так: \emph{<<У~одного очень хитрого мужчины, мистера А.~Л.~Г., есть неограниченный запас бусинок из $n$ цветов. Он хочет подарить как можно большему числу дам подарки-ожерелья из 6 бусин. Дамы заподозрят обман, если поймают мужчину на одинаковых ожерельях. Скольким дамам сможет сделать подарок мистер А.~Л.~Г.?>>}

    Ответом на данный вопрос, конечно же, является количество орбит группы диэдра $D_6$ на множестве всех раскрасок вершин правильно шестиугольника в $n$ цветов. А вы как думали!?

    Группа диэдра $D_6$ состоит из $12$ элементов: \begin{bullets}
        \item тождественного преобразования $e$; 
        \item двух поворотов $r^{\pm 1}$ на $\pm 60^\circ$;
        \item двух поворотов $r^{\pm 2}$ на $\pm 120^\circ$;
        \item центральной симметрии $r^3$;
        \item трёх отражений $s_{14}, s_{25}, s_{36}$ относительно больших диагоналей;
        \item и трёх отражений $\bar{s_{14}}, \bar{s_{25}}, \bar{s_{36}}$ относительно серединных перпендикуляров к сторонам.
    \end{bullets}
    Единица оставляет на месте все $n^6$ раскрасок. Раскраски, симметричные относительно других преобразований 
    %TODO: тут нужно вставить рисунок.
    Взяв на этих рисунках все допустимые сочетания цветов, получаем, соотвественно, $n, n^2, n^3, n^4$ и $n^3$ раскрасок. Тогда по \cref{th:burnside} число 6-бусинок равно \[\frac{n^6 + 3n^4 +4n^3+2n^2+2n}{12}.\]
\end{example}

