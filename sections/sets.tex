\section{Множества и отображения}
Этот параграф носит вспомогательный характер. В нём собраны некоторые факты о множествах и отображениях, используемые в этой книге.
\subsection{Множества}
В конце 19 века Георг Кантор создал теорию множеств, пытаясь строго обосновать математику. Его идеи казались парадоксальными (множество всех множеств!), но стали фундаментом. Мы используем ,,наивную`` версию, избегая парадоксов. Для
понимания этого курса достаточно интуитивного школьного представления о множестве как ,,абстрактной совокупности элементов произвольной природы``. Элементы множеств мы часто будем называть точками. Все точки в любом множестве, по определению, различны.

Множество $X$ задано, как только про любой объект можно сказать, является он элементом множества $X$ или нет. Принадлежность точки $x$ множеству $X$ записывается как $x\in X$. Два множества равны, если они состоят из одних и тех же элементов. Существует единственное множество, не содержащее ни одного элемента. Оно называется пустым и обозначается $\emptyset$. Если множество $X$ конечно, то мы обозначаем через $|X|$ количество точек в нём. 

Множество $X$ называется подмножеством множества $Y$, если каждый его элемент $x\in X$ лежит также и в $y \in Y$. В этом случае пишут $X \subset Y$ или $X \subseteq Y$. Отметим, что пустое множество является подмножеством любого множества и всякое множество является подмножеством самого себя. Подмножества, отличные от всего множества, называются собственными. В частности, пустое под-
множество непустого множества собственное. Если надо указать, что $X$ является собственным подмножеством в $Y$, используется обозначение $X \subsetneq Y$.

\begin{practice}
    Сколько всего подмножеств (включая пустое и несобственное) имеется у множества, состоящего из $n$ элементов.
\end{practice}

Для заданных множеств $X, Y$ их \emph{объединение} $X \cup Y$ состоит из всех элементов, принадлежащих хотя бы одному из множеств $X, Y$; \emph{пересечение} $X \cap Y$ состоит из всех элементов, принадлежащих одновременно каждому из множеств $X, Y$; \emph{разность} $X \setminus Y$ состоит из всех элементов множества $X$, которые не содержатся в $Y$.

\begin{definition}[Дизъюнкткное объединение]
    Множество $X$, явбляющееся объединением двух непересекающихся множеств $Y$ и $Z$, называется \emph{дизъюнктным объединением} $Y \sqcup Z$.
\end{definition}

\begin{example}
    Множество натуральных чисел является дизъюнтным объединением четных и нечетных чисел.

    Множество натуральных чисел является дизъюнтным объединением чисел с одним делителем (только 1), чисел с двумя делителями (простые) и чисел с большим количеством делителей (состовные).
\end{example}
\begin{example}
    Множество человеческих рук являюется дизъюнктным объединением правых и левых рук.
\end{example}

\begin{definition}[Декартово произведение]
    Множество $X \times Y$, элементами которого являются всевозможные пары $(x, y)$ с $x \in X,\; y \in Y$, называется \emph{декартовым произведением} множеств $X$ и $Y$.
\end{definition}

\begin{example}
    Координатная плоскость, которую не зря также называются декартовой, является декартовым произведением $\R \times \R$, такие произведения как и числа, удобно записывать в виде степени $\R^2$.
    
    Также как и на числовой прямой $\R$ можно выделить целые числа $\Z$, а говоря формально $\Z \subset \R$, то и $\Z^2 \coloneq \Z \times \Z \subset \R^2$, что является целочисленной решеткой на координатной плоскости.
\end{example}

\subsection{Отображения}
Отображения $f \colon X \to Y $ из множества $X$ в множество $Y$ есть правило, однозначно сопостовляющее каждой точке $x \in X$ некоторую точку $y = f(x) \in Y$, которая называется образом точки $x$ при отображении $f$. 

\begin{definition}[Полный прообраз]
    Множество всех таких точек $x \in X$, образ которых равен заданной точке $y \in Y$, называется \emph{полным прообразом} точки $y$ или \emph{слоем} отображения $f$ над $y$ и обозначается \[f^{-1}(y) \defeq \{x \in X \mid f(x) = y \} .\]
\end{definition}

\begin{remark}
    Полные прообразы различных точек не пересекаются и могут быть как пустыми, так и состоять из многих точек.
\end{remark}


\begin{definition}[Образ отображения ]
    Множетсво $y \in Y$, имеющих непустой прообраз, называется образом отображения $f$ и обозначается \[ f(X) = \im (f) \defeq \{y \in Y \mid \exists x \in X \colon f(x) = y \}. \]
\end{definition}

\begin{example}
    Отображение $f \colon \{1, 2, 3, 4 \} \to \{ A, B, C \}$, где $f(1) = A, f(2) = B, f(3) = A, f(4) = A$. Тогда, например, $f^{-1}(C) = \emptyset$, а $\im (f) = \{A, B \} \subset \{A, B, C \}$.
\end{example}

Два отображения $f, g \colon X \to Y$ равны, если $f(x) = g(x) $ для всех $x \in X$.

\begin{definition}[Сюръекция]
    Отображение $f \colon X \to Y$ называется \emph{наложением} (а также \emph{сюръекцией} или \emph{эпиморфизмом}), если $\im(f) = Y$. Мы будем отображать сюръективные отображения стрелками $f \colon X \twoheadrightarrow Y$. 
\end{definition}
\begin{definition}[Инъекция]
    Отображение $f \colon X \to Y$ называется \emph{вложение} (а также \emph{инъекцией} или \emph{мономорфизмом}), если $f(x_1) \neq f(x_2)$ при $x_1 \neq x_2$. Мы будем отображать сюръективные отображения стрелками $f \colon X \hookrightarrow Y$. 
\end{definition}

\begin{practice}
    Перечислите все отображения $\{0, 1, 2\} \to \{0, 1\}$ и $\{0, 1\} \to \{0, 1, 2\}$. Сколько среди них сюръекций и инъекций?
\end{practice}

\begin{definition}[Биекция]
    Отображение $f \colon X \to Y$, которое является и вложением, и наложемнием называется \emph{взаимно однозначным} (а также \emph{биекцией} или \emph{изоморфизмом}). Мы будем обозначать биекцию стрелками $X \simrightarrow Y$.
\end{definition}
\begin{practice}
    Из отображений \begin{inumerate}
        \item $\N \to \N \colon x \mapsto x^2$
        \item $\Z \to \Z \colon x \mapsto x^2$ 
        \item $\Z \to \Z \colon x \to 7x $
    \end{inumerate} выделите все вложения, наложения и биекции.

    
\end{practice}

\subsubsection{Слои отображения}
Задание отображения $f \colon X \to Y$ равносильно указания подмножества $\im(f) \subset Y$ и разбиению множества $X$ в дизъюнкткноe объединение занумерованных точками $y \in \im(f)$ непустых подмножества $f^{-1}(y)$:
\[
    \bigsqcup_{y \in \im (f)}f^{-1}(y).
\]

Такой взгляд на отображения часто оказывается полезным при подсчёте количества элементов в том, или ином множестве. Например, когда все непустые слои отображения $f \colon X \to Y$ состоят из одного и того же числа точек $a = |f^{-1}(y)|$, число элементов в образе отображения $f$ связано с числом элементов в множестве $X$ соотношением \[ |X| = a \cdot |\im(f) |.\]
Которое при всей своей простоте имеет много различных применений.

\subsection{Классы эквивалентности}
Альтернативным способом разбить множество $X$ в дизъюнкткноe объединение подмножеств состоит в том, чтобы объявить элементы, входящие в одно подмножество такого разбиения ,,эквивалентными``. Формализуем это так. 
\begin{definition}[Бинарное отношение]
    Назовём \emph{отношением} на множестве $X$ любое подмножество \[R \subset X\times X = \{(a, b)\mid a, b \in X\}.\]
\end{definition}

Принадлежность какой-то пары $(a, b)$ отношению $R$ обычно записывают как $a \sim_R b$, а еще чаще опускают до $a \sim b$.
\begin{example}
    На множестве целых чисел $\Z$ имеются отношения 
    \begin{center}
        \begin{tabular}{r l}
            равенство & $a \sim b \iff a =b $; \\
            неравенство & $a \sim b \iff a \leqslant b$ \\
            делимость & $a \sim b \iff a \text{ делится на } b$ \\
            сравнимость по модулю $n$ & $a \sim b \iff a \equiv b \pmod n$\footnote{это условие означает, что $a-b$ делится на $n$}.
        \end{tabular}
    \end{center}
\end{example}

\begin{example}[Отношение эквивалентности]
    Отношение $R \subset X \times X$ называется \emph{эквивалетностью}, если оно обладает тремя свойствами: 
    \begin{conditions}
    \item рефлексивность: $a \sim a, \quad \forall a \in X$;
    \item транзитивность: для любых $a, b, c \in X$ из $a \sim b$ и $b \sim c $ вытекает $a \sim c$;
    \item симметричность: $a \sim b \iff b \sim a, \quad \forall a, b \in X$. \end{conditions}
\end{example}
Если множество $X$ разбито в объединение непересекающихся подмножеств, то отношение $a \sim b$ означающее, что $a$ и $b$ лежат в одном и том же подмножестве этого разбиения, оче видно, является эквивалентностью.

Наоборот, пусть на множестве $X$ задано отношение эквивалентности $R$. Рассмотрим для каждого $x \in X$ подмножество в $X$, состоящее из всех элементов, эквивалентных $x$. Оно называется классом эквивалентности элемента $x$ и обозначается 
\[[x]_R \defeq \{z \in X \mid x \sim z\}.\]

Любые два класса $[x]_R$ и $[y]_R$ либо вообще не пересекаются, либо полностью совпадают. В самом деле, если существует элемент $z$, эквивалентный и $x$ и $y$, то в силу симметричности и транзитивности отношения $R$ элементы $x$ и $y$ будут эквивалентны между собой, а значит, любой элемент, эквивалентный $x$, будет эквивалентен также и $y$, и наоборот. Таким образом, множество $X$ распадается в дизъюнктное объединение различных классов эквивалентности.

\begin{definition}[Факторизация]
    Множество классов эквивалентности по отношению $R \subset X \times X$ называется \emph{фактором} множества $X$ по отношению $R$ и обозначается $X/R$. Наложение \[f \colon X \twoheadrightarrow X/R, \quad x \mapsto [x]_R,\]
    сопостовляющее каждому элементу $x \in X$ его класс эквивалентности называется \emph{отображением факторизации}.
\end{definition}

\begin{example}[Классы вычетов]
    Зафиксируем натуральное $n \in \N$. Фактор множества $\Z$ по отношению сравнимости по модулю $n$ обозначается $\Z/(n)$. Мы будем записывать его элементы символами $[z]_n$, где $z \in \Z$, и часто опускать индекс $n$. Класс эквивалентности \[[z]_n \coloneq \{x \in Z \mid (z -x ) \text{ делится на } n\}\] называется \emph{классом вычетов} по модулю $n$. Отображени факторизации \[f \colon \Z \twoheadrightarrow \Z/(n), \quad z \mapsto [z]_n\] называется \emph{приведением} по модулю $n$. Множество $\Z/(n)$ состоит из $n$ различных классов \[[0]_n ,[1]_n, \ldots [n-1]_n.\]

    При желании их можно воспринимать как остатки от деления на $n$, однако в практических вычислениях удобно работать именно с \emph{подмножествами}. Например, остаток от деления $12^{100}$ на $13$ можно искать как \[[12^{100}]_{13} = [12]_{13}^{100} = [-1]_{13}^{100} = [(-1)^{100}]_{13} = [1]_{13}.\]
\end{example}
