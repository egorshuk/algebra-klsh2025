\section{Множества и отображения}
Этот параграф носит вспомогательный характер. В нём собраны некоторые факты о множествах и отображениях, используемые в этой книге.
\subsection{Множества}
В наши цели не входит построение логически строгой теории множеств. Для
понимания этого курса достаточно интуитивного школьного представления о множестве как ,,абстрактной совокупности элементов произвольной природы``. Элементы множеств мы часто будем называть точками. Все точки в любом множестве, по определению, различны.

Множество $X$ задано, как только про любой объект можно сказать, является он элементом множества $X$ или нет. Принадлежность точки $x$ множеству $X$ записывается как $x\in X$. Два множества равны, если они состоят из одних и тех же элементов. Существует единственное множество, не содержащее ни одного элемента. Оно называется пустым и обозначается $\emptyset$. Если множество $X$ конечно, то мы обозначаем через $|X|$ количество точек в нём. 

Множество $X$ называется подмножеством множества $Y$, если каждый его элемент $x\in X$ лежит также и в $y \in Y$. В этом случае пишут $X \subset Y$ или $X \subseteq Y$. Отметим, что пустое множество является подмножеством любого множества и всякое множество является подмножеством самого себя. Подмножества, отличные от всего множества, называются собственными. В частности, пустое под-
множество непустого множества собственное. Если надо указать, что $X$ является собственным подмножеством в $Y$, используется обозначение $X \subsetneq Y$.

\begin{practice}
    Сколько всего подмножеств (включая пустое и несобственное) имеется у множества, состоящего из $n$ элементов.
\end{practice}

Для заданных множеств $X, Y$ их \emph{объединение} $X \cup Y$ состоит из всех элементов, принадлежащих хотя бы одному из множеств $X, Y$; \emph{пересечение} $X \cap Y$ состоит из всех элементов, принадлежащих одновременно каждому из множеств $X, Y$; \emph{разность} $X \setminus Y$ состоит из всех элементов множества $X$, которые не содержатся в $Y$.

\begin{definition}[Дизъюнктуное объединение]
    Множество $X$, явбляющееся объединением двух непересекающихся множеств $Y$ и $Z$, называется \emph{дизъюнктным объединением} $Y \sqcup Z$.
\end{definition}

\begin{example}
    Множество натуральных чисел является дизъюнтным объединением четных и нечетных чисел.

    Множество натуральных чисел является дизъюнтным объединением чисел с одним делителем (только 1), чисел с двумя делителями (простые) и чисел с большим количеством делителей (состовные).
\end{example}
\begin{example}
    Множество человеческих рук являюется дизъюнктным объединением правых и левых рук.
\end{example}

\begin{definition}[Декартово произведение]
    Множество $X \times Y$, элементами которого являются всевозможные пары $(x, y)$ с $x \in X,\; y \in Y$, называется \emph{декартовым произведением} множеств $X$ и $Y$.
\end{definition}

\begin{example}
    Координатная плоскость, которую не зря также называются декартовой, является декартовым произведением $\R \times \R$, такие произведения как и числа, удобно записывать в виде степени $\R^2$.
    
    Также как и на числовой прямой $\R$ можно выделить целые числа $\Z$, а говоря формально $\Z \subset \R$, то и $\Z^2 \coloneq \Z \times \Z \subset \R^2$, что является целочисленной решеткой на координатной плоскости.
\end{example}
