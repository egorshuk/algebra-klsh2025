\section{Действие группы на множестве}
\begin{definition}
    [Гомоморфизм групп]
    Отображение $\varphi\colon G_1 \to G_2$ называется \emph{гомоморфизмом}, если переводит композицию в композицию, иначе говоря для любых $a,b \in G_1$ в группе $G_2$ выполняется соотношене \(\varphi(ab) = \varphi(a)\varphi(b)\).

    \emph{Эпиморфизмом} называют сюръективный гомоморфизм, \emph{мономорфизмом} называют инъективный гомоморфизм, а \emph{изоморфизмом}, на самом деле, называют биективный гомоморфизм.
\end{definition}

\begin{practice}
    Докажите, что композиция двух гомоморфизмов --- тоже гомоморфизм.
\end{practice}

\begin{definition}
    [Образ гомоморфизма]
    Множество всех значений гомоморфизма $\varphi \colon G_1 \to G_2$ называется его \emph{образом} и обозначается $\im \varphi$ или $$\varphi (G_1) = \{\varphi(g_1)\mid g_1 \in G_1\}.$$
\end{definition}

Гомоморфизм переводит единицу в единицу, а обратный в обратный. Докажем только первое. \(\varphi(e_1)\varphi(e_1) = \varphi(e_1e_1)=\varphi(e_1)\), домножая на единицу равенство получаем то, что надо. Отсюда следует, что образ гомоморфизма --- подгруппа. \[\im \varphi \defeq \varphi(G_1) \subset G_2.\]

Напомню, что $\Aut(X)$, где $X$ --- множество, называется множество взаимнооднозначных отображений $X$ в себя, т.е. его перестановки.

\begin{definition}
    [Действие группы на множестве]
Гомоморфизм $\varphi\colon G \to \Aut(X)$ называется \emph{Действием} группы $G$ на множестве $X$ или \emph{представлением} группа $G$ автоморфизмами множества $X$. 

Отображение $\varphi(g) \colon X \to X$, отвечающее элементу $g$ при действии $\varphi$ удобно будет обозначать $\varphi_g \colon X \to X$. Тот факт, что $g \mapsto \varphi_g$ является гомоморфизмом групп означает, что $\varphi_{ab} = \varphi_a \circ \varphi_b$. Если хочется указать, что группа действует на множестве, то пишут $G \colon X$.
\end{definition}

Можно классифицировать действия по разному, приведу в пример классическую. Действие называется \emph{транзитивным}, если любую точку множества $X$ можно перевести в любую другую элементов из $G$. Действие называется \emph{свободным}, если каждое преобразование из $G$ действует на $X$ без неподвижных точек, т.е. $gx = x$ возможно лишь если $g = e$. Действие называется \emph{точным} или \emph{эффектиным}, если любое нетождественное преобразование из $G$ действует не тождественно, т.е. двигает что-то куда-то.

Разберём конкретные примеры действий группы, в первую очередь на себе.
\begin{example}
    [Регулярное действие]
    Пусть $X$ --- множество элементов группы $G$. Тогда \emph{левым регулярным дейсвтиям} называется гомоморфизм $\lambda_g: x \mapsto gx$, для выбраного элемента $g \in G$. Это действительно гомоморфизм групп: \[
        \lambda_{ab}(x) = abx = \lambda_a (bx) = \lambda_a \lambda_b (x).
    \]
    Так как равенство $gx = x$ в группе выполняется только при $g = e$, то такое действие свободно и в частности, эффективно.
    
    \emph{Правым регулярным действием} называется гомоморфизм $\rho_g: x \mapsto xg^{-1}$ правого умножения на обратный.\footnote{Тут $g^{-1}$, конечно, не с проста. Проверьте, что если бы было просто $g$, то преобразование было бы антигомоморфизмом.}

    \begin{practice}
        Убедитесь, что $\rho_g$ свободное действие.
    \end{practice}
\end{example}

Тем самым, любая абстрактная группа $G$ может быть реализована как группа преобразований некоторого множества. Например, левые регулярные представления числовых групп реализуют аддитивную группу $\R$ группой сдвигов $\lambda_v \colon x \mapsto x + v$ числовой прямой , а мультипликативную группу $\R^\times$ --- группой гомотетий $\lambda_c \colon x \mapsto cx$ проколотой прямой $\R^times = \R \setminus \{0\}$.

\begin{example}
    [Присоединённое действие]
    Отображение $Ad \colon G \to \Aut(G)$, сопоставляющее элементу $g\in G$ автоморфизм сопряжения этим элементом \[Ad_g \colon G \to G, \quad h \mapsto ghg^{-1},\] называется \emph{присоединённым} действием $G$ на себе.

    \begin{practice}
        Убедитесь, что $Ad_g$ является гомоморфизмом, а также и $Ad$.
    \end{practice}

    Образ присоединённого действия $\im Ad$ называется группой \emph{внутренних} автоморфизмов группы $G$ и обозначается $\Int(G)$. Автоморфизмы, которые не лежат в $\Int(G)$, называются, логично, \emph{внешними} автоморфизмами группы $G$. В отличие от левого и правого регулярных действий присоединённое действие, вообще говоря, не свободно и не точно. Заметим также, что если выполняется равенство $h = ghg^{-1}$, то выполняется и равенство $gh = hg$. Подгруппа элементов, которая удовлетворяет такому равенству называется \emph{центром} группы и обозначается \[Z(G) = \{g \in G \mid \forall h \in G\; gh=hg\}.\]
\end{example}

\subsection{Орбита и стабилизатор}
\subsubsection{Связь между орбитой и стабилизатором}
                                                                                                                                                                                                              

