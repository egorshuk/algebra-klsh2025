\section{Действие группы на множестве}
\begin{definition}
    [Гомоморфизм групп]
    Отображение $\varphi\colon G_1 \to G_2$ называется \emph{гомоморфизмом}, если переводит композицию в композицию, иначе говоря для любых $a,b \in G_1$ в группе $G_2$ выполняется соотношене \(\varphi(ab) = \varphi(a)\varphi(b)\).

    \emph{Эпиморфизмом} называют сюръективный гомоморфизм, \emph{мономорфизмом} называют инъективный гомоморфизм, а \emph{изоморфизмом}, на самом деле, называют биективный гомоморфизм.
\end{definition}

\begin{practice}
    Докажите, что композиция двух гомоморфизмов --- тоже гомоморфизм.
\end{practice}

\begin{definition}
    [Образ гомоморфизма]
    Множество всех значений гомоморфизма $\varphi \colon G_1 \to G_2$ называется его \emph{образом} и обозначается $\im \varphi$ или $$\varphi (G_1) = \{\varphi(g_1)\mid g_1 \in G_1\}.$$
\end{definition}

\begin{definition}
    [Слой отображения]
    Подмножество $f^{-1}(y) \subset X$ называется \emph{слоем отображения} $f \colon X \to Y$ над точкой $y \in Y$.
\end{definition}

Гомоморфизм переводит единицу в единицу, а обратный в обратный. Докажем только первое. \(\varphi(e_1)\varphi(e_1) = \varphi(e_1e_1)=\varphi(e_1)\), домножая на единицу равенство получаем то, что надо. Отсюда следует, что образ гомоморфизма --- подгруппа. \[\im \varphi \defeq \varphi(G_1) \subset G_2.\]

Напомню, что $\Aut(X)$, где $X$ --- множество, называется множество взаимнооднозначных отображений $X$ в себя, т.е. его перестановки.

\begin{definition}
    [Действие группы на множестве]
Гомоморфизм $\varphi\colon G \to \Aut(X)$ называется \emph{Действием} группы $G$ на множестве $X$ или \emph{представлением} группа $G$ автоморфизмами множества $X$. 

Отображение $\varphi(g) \colon X \to X$, отвечающее элементу $g$ при действии $\varphi$ удобно будет обозначать $\varphi_g \colon X \to X$. Тот факт, что $g \mapsto \varphi_g$ является гомоморфизмом групп означает, что $\varphi_{ab} = \varphi_a \circ \varphi_b$. Если хочется указать, что группа действует на множестве, то пишут $G \colon X$.
\end{definition}

Можно классифицировать действия по разному, приведу в пример классическую. Действие называется \emph{транзитивным}, если любую точку множества $X$ можно перевести в любую другую элементов из $G$. Действие называется \emph{свободным}, если каждое преобразование из $G$ действует на $X$ без неподвижных точек, т.е. $gx = x$ возможно лишь если $g = e$. Действие называется \emph{точным} или \emph{эффектиным}, если любое нетождественное преобразование из $G$ действует не тождественно, т.е. двигает что-то куда-то.

Разберём конкретные примеры действий группы, в первую очередь на себе.
\begin{example}
    [Регулярное действие]
    Пусть $X$ --- множество элементов группы $G$. Тогда \emph{левым регулярным дейсвтиям} называется гомоморфизм $\lambda_g: x \mapsto gx$, для выбраного элемента $g \in G$. Это действительно гомоморфизм групп: \[
        \lambda_{ab}(x) = abx = \lambda_a (bx) = \lambda_a \lambda_b (x).
    \]
    Так как равенство $gx = x$ в группе выполняется только при $g = e$, то такое действие свободно и в частности, эффективно.
    
    \emph{Правым регулярным действием} называется гомоморфизм $\rho_g: x \mapsto xg^{-1}$ правого умножения на обратный.\footnote{Тут $g^{-1}$, конечно, не с проста. Проверьте, что если бы было просто $g$, то преобразование было бы антигомоморфизмом.}

\end{example}
\begin{practice}
    Убедитесь, что $\rho_g$ свободное действие.
\end{practice}

Тем самым, любая абстрактная группа $G$ может быть реализована как группа преобразований некоторого множества. Например, левые регулярные представления числовых групп реализуют аддитивную группу $\R$ группой сдвигов $\lambda_v \colon x \mapsto x + v$ числовой прямой , а мультипликативную группу $\R^\times$ --- группой гомотетий $\lambda_c \colon x \mapsto cx$ проколотой прямой $\R^times = \R \setminus \{0\}$.

\begin{example}
    [Присоединённое действие]
    Отображение $Ad \colon G \to \Aut(G)$, сопоставляющее элементу $g\in G$ автоморфизм сопряжения этим элементом \[Ad_g \colon G \to G, \quad h \mapsto ghg^{-1},\] называется \emph{присоединённым} действием $G$ на себе.


    Образ присоединённого действия $\im Ad$ называется группой \emph{внутренних} автоморфизмов группы $G$ и обозначается $\Int(G)$. Автоморфизмы, которые не лежат в $\Int(G)$, называются, логично, \emph{внешними} автоморфизмами группы $G$. В отличие от левого и правого регулярных действий присоединённое действие, вообще говоря, не свободно и не точно. Заметим также, что если выполняется равенство $h = ghg^{-1}$, то выполняется и равенство $gh = hg$. Подгруппа элементов, которая удовлетворяет такому равенству называется \emph{центром} группы и обозначается \[Z(G) = \{g \in G \mid \forall hj \in G\; gh=hg\}.\]
\end{example}
\begin{practice}
    Убедитесь, что $Ad_g$ является гомоморфизмом, а также и $Ad$.
\end{practice}

\subsection{Орбита и стабилизатор}
Со всякой группой преобразований $G$ множества $X$ связано отношение $x \sim y$ на $X$, означающее, что $y = gx$ для некоторого преобразование $g \in G$. Проверим некоторые важные свойства этого отношения: \begin{conditions} \item \emph{рефлексивность:} $x = ex \implies x \sim x$;
    \item \emph{симметричность:} $y = gx \implies x = g^{-1}y$;
    \item \emph{транзитивность:} $y = gx,\quad z = hy \implies z = ghy$.
\end{conditions}

Такие отношения называются отношениями \emph{эквивалентности}, действительно, логично считать, что точки которые, совмещаются преобразованием из $G$ в каком-то смысле эквиваленты. \emph{Классом эквивалентсости} точки $X$ называются все точки, которые ей эквиваленты. 
\begin{definition}[Орбита]
    Классом эквивалентности отношения $x \sim y \iff gx = y$ называется \emph{орбитой} точки $x$ под действием $G$ и обозначается 
    \[Gx = \{gx \mid g \in G\}.\]

    Множеством всех орбит называется \emph{фактором} множества $X$ под действием $G$ и обозначается $X/G$.
\end{definition}

\begin{proposition}
    Множество $X$ распадается в объединение орбит, в том смысле, что две орбиты либо полностью совпадают, либо вообще не пересекаются. 
\end{proposition}
\begin{proof}
    В самом деле, возьмём две орбиты $Gx$ и $Gy$, а также $z$, который принадлежит и первой и второй орбите: \[g_1x = z, \quad g_2y = z.\]
    В таком случае, устанавливаем, что $$g_1x = g_2y \implies x = g_1^{-1}g_2y \implies y \in G_x.$$
    Тем самым $G_y \subset G_x$, аналогично $G_x \subset G_y$.
\end{proof}

С каждой точкой $x \in X$ и его орбитой $Gx$ связано сюръективное отображение $\ev_x \colon G \twoheadrightarrow Gx,\; g \mapsto gx$, слой которого над точкой $y \in Gx$ состоит из всех преобразований группы $G$, переводящих $x$ в $y$. 
\begin{definition}
    [Транспортёр]
    Слой отображения $\ev_x \colon G \twoheadrightarrow Gx$ над точкой $y$ называется \emph{трансортёром} $x$ в $y$ и обозначается \[G_{xy} = \{g \in G \mid gx = y\}.\]
\end{definition}
\begin{definition}[Стабилизатор]
    Слоем отображения $\ev_x \colon G \twoheadrightarrow Gx$ над самой точкой $x$ называется \emph{стабилизатором} точки $x \in X$ и обозначается \[ \Stab(x) = \{g \in G \mid gx = x \} = G_{xx}.\]
\end{definition}

\begin{lemma}\label{lem:orb_trans}
    Для любых $x, y, z$ из одной орбиты имеются взаимнообратные биекции:
    \begin{center}
        \begin{tikzcd}[column sep = large]
            \Stab(x) \arrow[bend left]{r}{s\mapsto hsg^{-1}} & G_{yz} \arrow[bend left]{l}{f \mapsto h^{-1}fg}
        \end{tikzcd}
    \end{center}
\end{lemma}
\begin{proof}
    Если $y = gx$ и $z = hx$, то $$sg^{-1}y = \underbrace{sx}_{x} = h^{-1}z \implies hsg^{-1} \in G_{yz}, $$
    для всех $s \in \Stab(x)$. Наоборот, если $fy = z$, то \[
        (h^{-1}fg)x = (h^{-1}f)y = {h^{-1}}z = x \implies h^{-1}fg \in \Stab(x).
    \]
\end{proof}

\begin{proposition}
    Стабилизаторы всех точек из одной орбиты равномощны и сопряжены: \[
    y = gx \implies \Stab(y) = g \Stab(x) g^{-1} = \{ ghg^{-1} \mid h \in \Stab(y) \}.\]
    \begin{proof}
        По \cref{lem:orb_trans}, если $z = y$, а $h=g$, следует это утверждение.
    \end{proof}
\end{proposition}
\subsubsection{Связь между орбитой и стабилизатором}

\begin{proposition}[Формула для длины орбиты]
Длина орбиты произвольной точки $x \in X$ при действии на неё конечной группы преобразований $G$ равна $$|Gx| = \frac{|G|}{|\Stab(x)}.$$ В частности, длины всех орбит и порядки стабилизаторов всех точек являются делителем порядка группы.
\end{proposition}
\begin{proof}
    Группа $G$ является объединением непересекающихся множеств $G_{yx}$ по всем $y \in Gx$. Тогда по \cref{lem:orb_trans} все эти множества состоят из $|\Stab(x)|$ элементов. Получаем, что $|Gx|\cdot|\Stab(x)|=|G|.$
\end{proof}


