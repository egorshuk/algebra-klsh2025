\section{Введение в группы}

Алгебра -- наука о структурах, которые описываются с помозью операций и законов. 
Группы -- это один из основных объектов алгебры. Это самое ``базовое понятие'', 
но оно же и является центральным.

\begin{definition}
    [Группа]
    Это множество $G$ с операцией $\star$, которое обладает следующими свойствами: 
    \begin{conditions}
        \item \textit{Замкнутость}: $$\forall a, b \in G: a \star b \in G.$$
        \item \textit{Ассоциативность}: $$\forall a, b, c \in G: (a \star b) \star c = a \star (b \star c).$$
        \item \textit{Наличие нейтрального элемента}: $$\exists e \in G: \forall a \in G: e \star a = a \star e = a.$$
        \item \textit{Наличие обратного элемента}: $$\forall a \in G: \exists b \in G: a \star b = b \star a = e.$$
    \end{conditions}
    Для группы также существует обозначение: $(G, \star).$
\end{definition}

Существуют различные классификации групп. Например, классификация по типу операции. 
Бывают группы по сложению (аддитивные), то есть с операцией сложения. 
А также бывают группы по умножению (мультипликативные) -- с операцией умножения.

\begin{example}
        $(\Z, +)$ множество целых чисел с операцией сложения.
\end{example}
\begin{example}
    $({\Z}/{(5)}, +)$ множество остатков по модулю 5 с операцией сложения.
\end{example}
\begin{example}
    Как множество -- движения правильной фигуры, а операция тут -- композиция этих движений.
\end{example}

\begin{definition}[Абелева группа]
   % Ну напиши определенеи абелевой группы
    Группа $G$\footnote{Часто операция опускается и подразумевается, что группа мультипликативна} 
    называется абелевой, если она коммутативна, то есть: \[
        \forall a, b \in G: ab = ba.
    .\] 
\end{definition}

В этом моменте нужно себя спросить: \emph{``А что, бывает по-другому?!''} И вот оказывается, что бывает.
Для этого, можно рассмотреть один яркий пример. Представьте, что в вашей группе находятся два элемента: 
\begin{example}
    Пусть у нас есть группа $G$, которая содержит в себе, по крайней мере два элемента: 
    $a = \text{``надеть носок''}$ и $b = \text{``надеть ботинок''}$.
    Тогда одна последовательность действий не приведет к \emph{странным взглядам окружающих,} а другая да.
\end{example}

\begin{practice}
   Какой из этих случаев ``нормален'', а какой нет? 
\end{practice}
